\documentclass{article}

\usepackage[utf8]{inputenc}
\usepackage[margin=1.0in]{geometry}
\usepackage{color}
\usepackage{amsmath}

\begin{document}
\begin{center}
  \textbf{Weakly Compressible AB Diblock and a C-Grafted Nanorod with
  Smearing Derivation}
\end{center}

\section{Canonical Ensemble Derivation (with $w_+$ and $w_{AC}^{(\pm)}$ fields)}

The system is composed of $n_D$ A\textbackslash B diblock chains and an explicit, neutral
  nanoparticle with $n_G$ grafted chains of length $N$.
Each diblock chain has $P_A + P_B = P$ segments.
$\chi_{IJ}$ is the interaction strength between components $I$ and $J$
  $AB$, $AC$, $BC$.
Segment center densities are defined as
\begin{align*}
  \hat{\rho}_{DA,c} (\mathbf{r}) =&
    \sum_{i=1}^{n_D} \sum_{j=1}^{P_A}
    \delta(\mathbf{r} - \mathbf{r}_{i,j}) \\
  \hat{\rho}_{DB,c} (\mathbf{r}) =&
    \sum_{i=1}^{n_D} \sum_{j=P_A+1}^{P}
    \delta(\mathbf{r} - \mathbf{r}_{i,j}) \\
  \hat{\rho}_{GC,c} (\mathbf{r}) =&
    \sum_{i=1}^{n_G} \sum_{j=1}^{N}
    \delta(\mathbf{r} - \mathbf{r}_i)
\end{align*}
For all of the polymer segments, the full (smeared) segment densities are given
  by:
\begin{align*}
  \hat{\rho}_K(\mathbf{r}) = (h \ast \hat{\rho}_{K,c})(\mathbf{r})
\end{align*}
where $K \in \{ DA, DB, GC\}$ and $h$ is the segment density distribution
  function given by the Gaussian
\begin{align*}
  h(\mathbf{r}) = \left( \frac{1}{2\pi a^2} \right)^{d/2}
  \exp \left( - \frac{|\mathbf{r}|^2}{2a^2}  \right)
\end{align*}
where $a$ is the segment size and $d$ is the number of dimensions.
The nanoparticle density is given by
\begin{align*}
  \hat{\rho}_P(\mathbf{r}) =
    \frac{\rho_0}{4}
    \textrm{erfc} \left(
      \frac{|\mathbf{u} \cdot (\mathbf{r} - \mathbf{r}_c)| - L_P}{\xi} 
    \right)
    \textrm{erfc} \left(
      \frac{|\mathbf{u} \times (\mathbf{r} - \mathbf{r}_c)| - R_P/2}{\xi}       
    \right)
\end{align*}
where $R_P$ is the nanoparticle radius, $L_P$ is the nanoparticle length, $\rho_0$ is the bulk density,
  and $\xi$ controls the nanoparticle interface width.
The graft site distribution is defined as a thin shell at the surface of the
  nanoparticle defined as
\begin{align*}
  \Gamma_\sigma(\mathbf{r}) =&
    \frac{| \nabla \Gamma(\mathbf{r}) |}
         {\int d \mathbf{r^\prime} | \nabla \Gamma(\mathbf{r^\prime}) | }
\end{align*}
Note that with this definition of $\Gamma_\sigma$, it's important to make sure
  to use odd numbers of grid points in each dimension to avoid Nyquist mode
  problems.
The harmonic bond potential between connected segments is given by
\begin{align*}
  \beta U_0 =
    \sum_{i=1}^{n_D} \sum_{j=1}^{P-1}
    \frac{3 \left| \mathbf{r}_{i,j+1} - \mathbf{r}_{i,j} \right| ^ 2 }
         { 2 b^2 }
    +
    \sum_{i=1}^{n_G} \sum_{j=1}^{N-1}
    \frac{3 \left| \mathbf{r}_{i,j+1} - \mathbf{r}_{i,j} \right| ^ 2 }
         { 2 b^2 }
    % +
    % \underbrace{\sum_{i=1}^{n_G}  \frac{3}{2b^2} 
    % \vert \mathbf{r}_{i,1} - \mathbf{r}_{i,\perp} \vert^2}_
    % {\substack{\text{contribution from bonding } \\
    % \text{the chain  to the ghost site }}}
\end{align*}
The nonbonded interaction potential is given by
\begin{align*}
  \beta U_1 =&
    \frac{\chi_{AB}}{\rho_0} \int d\mathbf{r} \hat{\rho}_{DA} \hat{\rho}_{DB}
    + \frac{-\chi_{AC}}{\rho_0} \int d\mathbf{r} \hat{\rho}_{DA} \hat{\rho}_{GC}
    + \frac{\chi_{BC}}{\rho_0} \int d\mathbf{r} \hat{\rho}_{DB} \hat{\rho}_{GC}
    \\
    &+ \frac{\chi_{AP}}{\rho_0} \int d\mathbf{r} \hat{\rho}_{DA} \hat{\rho}_P
    + \frac{\chi_{BP}}{\rho_0} \int d\mathbf{r} \hat{\rho}_{DB} \hat{\rho}_P
    + \frac{\chi_{CP}}{\rho_0} \int d\mathbf{r} \hat{\rho}_{GC} \hat{\rho}_P \\
  =&
    \sum_{IJ, I \ne J} \frac{s_{IJ}\chi_{IJ}}{\rho_0}
    \int d \mathbf{r} \hat{\rho}_I \hat{\rho}_J
\end{align*}
where all $\chi_{IJ} > 0$ and $s_{IJ}$ is either +1 (repulsive interaction) or
  -1 (attractive interaction).
In the specific case we'll cover here, with a neutral particle, grafts that
  are attracted to $A$ and repulsed by $B$, $s_{AC} = -1$ and
  $s_{AB} = s_{BC} = 1$, and $\chi_{IP} = 0$ for all $I$.
A Helfand incompressibility potential penalizes deviations away from $\rho_0$,
  and is given by
\begin{align*}
  \beta U_2 = \frac{\kappa}{2 \rho_0} \int d \mathbf{r}
    \left[ \hat{\rho}_+ (\mathbf{r}) - \rho_0 \right] ^ 2
\end{align*}
where
\begin{align*}
  \hat{\rho}_+ =
    \hat{\rho}_{DA} + \hat{\rho}_{DB} + \hat{\rho}_{GC} + \hat{\rho}_P
\end{align*}
  is the local total density.
This gives us a canonical partition function of
\begin{align*}
  Z_C = \frac{1}{n_D!n_G! \left( \lambda_T^d \right)^{n_D+n_G}}
    \int d \mathbf{r}^{n_DP} \int d \mathbf{r}^{n_GN}
    \exp \left( -\beta U_0 - \beta U_1 - \beta U_2 \right)
\end{align*}
where $d$ represents the dimensionality of the system.

To prepare this for a particle-to-field transformation, let's define
\begin{align*}
  \hat{\rho}_{AB}^{(\pm)} (\mathbf{r}) =&
    \hat{\rho}_{DA}(\mathbf{r}) \pm \hat{\rho}_{DB}(\mathbf{r}) \\
  \hat{\rho}_{AC}^{(\pm)} (\mathbf{r}) =&
    \hat{\rho}_{DA}(\mathbf{r}) \pm \hat{\rho}_{GC}(\mathbf{r}) \\
  \hat{\rho}_{BC}^{(\pm)} (\mathbf{r}) =&
    \hat{\rho}_{DB}(\mathbf{r}) \pm \hat{\rho}_{GC}(\mathbf{r})
\end{align*}
With these definitions, we can rewrite $\beta U_1$ as
\begin{align*}
  \beta U_1 =&
    \frac{\chi_{AB}}{4\rho_0}
    \int d \mathbf{r}
    \left(
      \hat{\rho}_{AB}^{(+)}(\mathbf{r})^2
      - \hat{\rho}_{AB}^{(-)}(\mathbf{r})^2
    \right)
    +
    \frac{-\chi_{AC}}{4\rho_0}
    \int d \mathbf{r}
    \left(
      \hat{\rho}_{AC}^{(+)}(\mathbf{r})^2
      - \hat{\rho}_{AC}^{(-)}(\mathbf{r})^2
    \right)
    +
    \frac{\chi_{BC}}{4\rho_0}
    \int d \mathbf{r}
    \left(
      \hat{\rho}_{BC}^{(+)}(\mathbf{r})^2
      - \hat{\rho}_{BC}^{(-)}(\mathbf{r})^2
    \right) \\
    &+
    \frac{\chi_{AP}}{\rho_0} \int d\mathbf{r} \hat{\rho}_{DA} \hat{\rho}_P
    +
    \frac{\chi_{BP}}{\rho_0} \int d\mathbf{r} \hat{\rho}_{DB} \hat{\rho}_P
    +
    \frac{\chi_{CP}}{\rho_0} \int d\mathbf{r} \hat{\rho}_{GC} \hat{\rho}_P
\end{align*}
From there, using the Gaussian functional integral (See Fredrickson eqns C.27
  and C.28.), we get
\begin{align*}
  \exp(-\beta U_1) =&
    \frac{1}{
      \Omega_{AB}^{(+)} \Omega_{AB}^{(-)}
      \Omega_{AC}^{(+)} \Omega_{AC}^{(-)}
      \Omega_{BC}^{(+)} \Omega_{BC}^{(-)}
    }
    \int \mathcal{D} w_{AB}^{(+)} \int \mathcal{D} w_{AB}^{(-)}
    \int \mathcal{D} w_{AC}^{(+)} \int \mathcal{D} w_{AC}^{(-)}
    \int \mathcal{D} w_{BC}^{(+)} \int \mathcal{D} w_{BC}^{(-)} \\
    &\times
    \exp \left(
      - \frac{\rho_0}{\chi_{AB}} \int d \mathbf{r} w_{AB}^{(+)}(\mathbf{r})^2
      - i\int d \mathbf{r} \hat{\rho}_{AB}^{(+)}(\mathbf{r}) w_{AB}^{(+)}(\mathbf{r})
    \right)
    \exp \left(
      - \frac{\rho_0}{\chi_{AB}} \int d \mathbf{r} w_{AB}^{(-)}(\mathbf{r})^2
      + \int d \mathbf{r} \hat{\rho}_{AB}^{(-)}(\mathbf{r}) w_{AB}^{(-)}(\mathbf{r})
    \right) \\
    &\times
    \exp \left(
      - \frac{\rho_0}{\chi_{AC}} \int d \mathbf{r} w_{AC}^{(+)}(\mathbf{r})^2
      + \int d \mathbf{r} \hat{\rho}_{AC}^{(+)}(\mathbf{r}) w_{AC}^{(+)}(\mathbf{r})
    \right)
    \exp \left(
      - \frac{\rho_0}{\chi_{AC}} \int d \mathbf{r} w_{AC}^{(-)}(\mathbf{r})^2
      - i\int d \mathbf{r} \hat{\rho}_{AC}^{(-)}(\mathbf{r}) w_{AC}^{(-)}(\mathbf{r})
    \right) \\
    &\times
    \exp \left(
      - \frac{\rho_0}{\chi_{BC}} \int d \mathbf{r} w_{BC}^{(+)}(\mathbf{r})^2
      - i\int d \mathbf{r} \hat{\rho}_{BC}^{(+)}(\mathbf{r}) w_{BC}^{(+)}(\mathbf{r})
    \right)
    \exp \left(
      - \frac{\rho_0}{\chi_{BC}} \int d \mathbf{r} w_{BC}^{(-)}(\mathbf{r})^2
      + \int d \mathbf{r} \hat{\rho}_{BC}^{(-)}(\mathbf{r}) w_{BC}^{(-)}(\mathbf{r})
    \right) \\
    &\times
    \exp \left(
      - \frac{\chi_{AP}}{\rho_0} \int d \mathbf{r} \hat{\rho}_{DA} \hat{\rho}_P
    \right)
    \exp \left(
      - \frac{\chi_{BP}}{\rho_0} \int d \mathbf{r} \hat{\rho}_{DB} \hat{\rho}_P
    \right)
    \exp \left(
      - \frac{\chi_{CP}}{\rho_0} \int d \mathbf{r} \hat{\rho}_{GC} \hat{\rho}_P
    \right).
\end{align*}
and
\begin{align*}
  \exp(- \beta U_2) =&
    \frac{1}{\Omega_+}
    \int \mathcal{D} w_+
    \exp \left(
      -\frac{\rho_0}{2\kappa}
      \int d \mathbf{r} w_+(\mathbf{r})^2
      + i\int d \mathbf{r} (\rho_0 - \hat{\rho}_+(\mathbf{r})) w_+(\mathbf{r})
    \right) \\
\end{align*}
The $\exp (-\beta U_1)$ equation can generalize to
\begin{align*}
  \exp (-\beta U_1) =&
    \prod_{IJ, I \ne J, I \ne P} \left(
      \frac{1}{\Omega_{IJ}^{(+)}\Omega_{IJ}^{(-)}}
      \int \mathcal{D} w_{IJ}^{(+)} \int \mathcal{D} w_{IJ}^{(-)}
    \right)
    \exp \left(
      -\frac{\rho_0}{\chi_{IJ}}
      \int d \mathbf{r} \left( 
        w_{IJ}^{(+)}(\mathbf{r})^2 + w_{IJ}^{(-)}(\mathbf{r})^2
      \right)
    \right) \\
    &\times
    \exp \left(
      \int d \mathbf{r} \left[ 
        - i \hat{\rho}_{IJ}^{(s_{IJ})} w_{IJ}^{(s_{IJ})}
        + \hat{\rho}_{IJ}^{(-s_{IJ})} w_{IJ}^{(-s_{IJ})}
      \right]
    \right) \\
    &\times
    \prod_{I, I \ne P} \exp \left(
      - \frac{s_{IP}\chi_{IP}}{\rho_0}
      \int d \mathbf{r} \hat{\rho}_I \hat{\rho}_P
    \right)
\end{align*}
where, for example, $w_{IJ}^{(s_{IJ})}$ is $w_{IJ}^{(+)}$ if $s_{IJ} = +1$ and
  $w_{IJ}^{(-)}$ if $s_{IJ} = -1$.
With this notation, $w_{IJ}^{(-s_{IJ})}$ is $w_{IJ}^{(-)}$ if $s_{IJ} = +1$ and
  $w_{IJ}^{(+)}$ if $s_{IJ} = -1$.
Before moving on, to make our lives easier, let's define
\begin{align*}
  \Omega =
    \Omega_+
    \Omega_{AB}^{(+)} \Omega_{AB}^{(-)}
    \Omega_{AC}^{(+)} \Omega_{AC}^{(-)}
    \Omega_{BC}^{(+)} \Omega_{BC}^{(-)}
\end{align*}
Now the canonical partition function looks like
\begin{align*}
  Z_C =& \frac{1}{n_D!n_G! \left( \lambda_T^d \right)^{n_D+n_G}}
    \frac{1}{\Omega}
    \int \hdots \int \mathcal{D} \{w\}
    \int d \mathbf{r}^{n_D P + n_G N} \\
    &\times
    \exp \left(
      - \frac{\rho_0}{2\kappa} \int d \mathbf{r} w_+(\mathbf{r})^2
      + i \rho_0 \int d\mathbf{r} w_+
      + \sum_{IJ,I \ne J, I \ne P}
      - \frac{\rho_0}{\chi_{IJ}}
      \int d \mathbf{r}
      \left(
        w_{IJ}^{(+)} (\mathbf{r})^2 + w_{IJ}^{(-)} (\mathbf{r})^2
      \right)
    \right) \\
    &\times
    \exp \left(
      \sum_{IP,I \ne P}
      - \frac{s_{IP}\chi_{IP}}{\rho_0}
      \int d \mathbf{r} \hat{\rho}_I \hat{\rho}_P
      - i\int d \mathbf{r} w_+ \hat{\rho}_+
      % s
    \right) \\
    &\times
    \exp \left(
      - \sum_{i=1}^{n_D} \sum_{j=1}^{P-1}
      \frac{3 \left| \mathbf{r}_{i,j+1} - \mathbf{r}_{i,j} \right| ^ 2 }
           { 2 b^2 }
      - \sum_{i=1}^{n_G} \sum_{j=1}^{N-1}
      \frac{3 \left| \mathbf{r}_{i,j+1} - \mathbf{r}_{i,j} \right| ^ 2 }
           { 2 b^2 } 
      % -
      % \sum_{i=1}^{n_G}  \frac{3\vert \mathbf{r}_{i,1} - \mathbf{r}_{i,\perp} \vert^2} %numerator
      % {2b^2}  % denominator
    \right) \\
    &\times
    \exp \left(
      \sum_{IJ, I \ne J, I \ne P}
      \int d \mathbf{r} \left(
        -i \hat{\rho}_{IJ}^{(s_{IJ})} w_{IJ}^{(s_{IJ})}
        + \hat{\rho}_{IJ}^{(-s_{IJ})} w_{IJ}^{(-s_{IJ})}
      \right)
    \right)
\end{align*}
Let's now rearrange all the terms with any $\hat{\rho}$ in them to determine
  what each $w_I$ is:
\begin{align*}
  \hat{\rho}\textrm{ Terms} =
    &\exp \left(
      \int d \mathbf{r} \left[
        - i w_+ \hat{\rho}_+
        -
        \sum_{IP,I \ne P}
        \frac{s_{IP}\chi_{IJ}}{\rho_0}
        \hat{\rho}_I \hat{\rho}_P
        +
        \sum_{IJ, I \ne J, I \ne P}
        \left(
          -i \hat{\rho}_{IJ}^{(s_{IJ})} w_{IJ}^{(s_{IJ})}
          + \hat{\rho}_{IJ}^{(-s_{IJ})} w_{IJ}^{(-s_{IJ})}
        \right)
      \right]
    \right) \\
  =
    &\exp \left(
      \int d \mathbf{r} \left[
        - i w_+ \hat{\rho}_{DA}
        +
        - \frac{s_{AP}\chi_{AP}}{\rho_0}
        \hat{\rho}_{DA} \hat{\rho}_P
        +
        \left(
          -i \hat{\rho}_{DA} w_{AB}^{(+)} + \hat{\rho}_{DA} w_{AB}^{(-)}
          -i \hat{\rho}_{DA} w_{AC}^{(-)} + \hat{\rho}_{DA} w_{AC}^{(+)}
        \right)
      \right]
    \right) \\
    &\times
    \exp \left(
      \int d \mathbf{r} \left[
        - i w_+ \hat{\rho}_{DB}
        -
        \frac{s_{BP}\chi_{BP}}{\rho_0}
        \hat{\rho}_{DB} \hat{\rho}_P
        +
        \left(
          -i \hat{\rho}_{DB} w_{AB}^{(+)} - \hat{\rho}_{DB} w_{AB}^{(-)}
          -i \hat{\rho}_{DB} w_{BC}^{(+)} + \hat{\rho}_{DB} w_{BC}^{(-)}
        \right)
      \right]
    \right) \\
    &\times
    \exp \left(
      \int d \mathbf{r} \left[
        - i w_+ \hat{\rho}_{GC}
        -
        \frac{s_{CP}\chi_{CP}}{\rho_0}
        \hat{\rho}_{GC} \hat{\rho}_P
        +
        \left(
          +i \hat{\rho}_{GC} w_{AC}^{(-)} + \hat{\rho}_{GC} w_{AC}^{(+)}
          -i \hat{\rho}_{GC} w_{BC}^{(+)} - \hat{\rho}_{GC} w_{BC}^{(-)}
        \right)
      \right]
    \right) \\
            &\times
      \exp\left(
        - \int d \mathbf{r} \hat{\rho}_{P} w_P %possibly? 
      \right)
\\
  =
    &\exp \left(
      \int d \mathbf{r} \hat{\rho}_{DA} \left[
        - i w_+ 
        -
        \frac{s_{AP}\chi_{AP}}{\rho_0}
        \hat{\rho}_P
        +
        \left(
          -i w_{AB}^{(+)} + w_{AB}^{(-)}
          -i w_{AC}^{(-)} + w_{AC}^{(+)}
        \right)
      \right]
    \right) \\
    &\times
    \exp \left(
      \int d \mathbf{r} \hat{\rho}_{DB} \left[
        - i w_+
        -
        \frac{s_{BP}\chi_{BP}}{\rho_0}
        \hat{\rho}_P
        +
        \left(
          -i w_{AB}^{(+)} - w_{AB}^{(-)}
          -i w_{BC}^{(+)} + w_{BC}^{(-)}
        \right)
      \right]
    \right) \\
    &\times
    \exp \left(
      \int d \mathbf{r} \hat{\rho}_{GC} \left[
        - i w_+
        -
        \frac{s_{CP}\chi_{CP}}{\rho_0}
        \hat{\rho}_P
        +
        \left(
          +i w_{AC}^{(-)} + w_{AC}^{(+)}
          -i w_{BC}^{(+)} - w_{BC}^{(-)}
        \right)
      \right]
    \right) \\
            &\times
      \exp\left(
        - \int d \mathbf{r} \hat{\rho}_{P} w_P %possibly? 
      \right)
\\
  =
    &\exp \left(
      - \int d \mathbf{r} \hat{\rho}_{DA} w_A
      - \int d \mathbf{r} \hat{\rho}_{DB} w_B
      - \int d \mathbf{r} \hat{\rho}_{GC} w_C
       - \int d \mathbf{r} \hat{\rho}_{P} w_P %possibly? 
    \right) \\
\end{align*}
where
\begin{align*}
  w_A &=\;
    i \left( w_+ + w_{AB}^{(+)} + w_{AC}^{(-)} \right)
    - w_{AB}^{(-)} - w_{AC}^{(+)}
    + \frac{s_{AP}\chi_{AP}}{\rho_0} \hat{\rho}_P \\
  w_B &=\;
    i \left( w_+ + w_{AB}^{(+)} + w_{BC}^{(+)} \right)
    + w_{AB}^{(-)} - w_{BC}^{(-)}
    + \frac{s_{BP}\chi_{BP}}{\rho_0} \hat{\rho}_P \\
  w_C &=\;
    i \left( w_+ - w_{AC}^{(-)} + w_{BC}^{(+)} \right)
    - w_{AC}^{(+)} + w_{BC}^{(-)}
    + \frac{s_{CP}\chi_{CP}}{\rho_0} \hat{\rho}_P \\
\end{align*}
% Looking at these field definitions, we can generalize with a rule-based process.
% Here's how we could construct any $w_X$ where X is A, B, or C; J and K
%   are the other two of A, B, and C but K $\ne$ X:

% \begin{enumerate}
%   \item Start with
%   {
%     \color{green}  %verified by Chris and Ben
%     \begin{align*}
%       w_X &=\;
%         i \left( w_+ + w_{XJ}^{(+)} + w_{XK}^{(+)} \right)
%         - w_{XJ}^{(-)} - w_{XJ}^{(-)}
%         + \frac{s_{XP}\chi_{XP}}{\rho_0} \hat{\rho}_P \\
%     \end{align*}
%   }
%   Note that, for example if X is B and J is A, then $w_{XJ}$ is actually $w_{AB}$,
%     not $w_{BA}$.
%   As an example, let's go through the process for $w_C$.
%   We start with:
%   {
%     \color{green}  %verified by Chris and Ben
%     \begin{align*}
%       w_C &=\;
%         i \left( w_+ + w_{AC}^{(+)} + w_{BC}^{(+)} \right)
%         - w_{AC}^{(-)} - w_{BC}^{(-)}
%     + \frac{s_{CP}\chi_{CP}}{\rho_0} \hat{\rho}_P \\
%     \end{align*}
%   }

%   \item For any $XJ$ where $\chi_{XJ}$ is attractive ($s_{XJ}$ = -1), change any
%     $(+)$ to a $(-)$ and vice versa.
%   This turns $w_C$ into:
%   {
%   \color{green}  %verified by Chris and Ben
%     \begin{align*}
%       w_C &=\;
%         i \left( w_+ + w_{AC}^{(-)} + w_{BC}^{(+)} \right)
%         - w_{AC}^{(+)} - w_{BC}^{(-)}
%     + \frac{s_{CP}\chi_{CP}}{\rho_0} \hat{\rho}_P \\
%     \end{align*}
%   }

%   \item For any $w_{XJ}^{(-)}$ where $X > J$, multiply it by $-1$ (flip the sign).
%   This gives us our final $w_C$:  
%   {
%     \color{green}
%     \begin{align*}
%       w_C &=\;
%         i \left( w_+ - w_{AC}^{(-)} + w_{BC}^{(+)} \right)
%         - w_{AC}^{(+)} + w_{BC}^{(-)}
%     + \frac{s_{CP}\chi_{CP}}{\rho_0} \hat{\rho}_P \\
%     \end{align*}
%   }
% \end{enumerate}

% ============= IGNORE EVERYTHING BELOW HERE ==============

Using the definitions of $\hat{\rho}_{IJ}^{(\pm)}$, and defining
\begin{align*}
  w_A &=\;
    i \left( w_+ + w_{AB}^{(+)} + w_{AC}^{(-)} \right)
    - w_{AB}^{(-)} - w_{AC}^{(+)}
    + \frac{s_{AP}\chi_{AP}}{\rho_0} \hat{\rho}_P \\
  w_B &=\;
    i \left( w_+ + w_{AB}^{(+)} + w_{BC}^{(+)} \right)
    + w_{AB}^{(-)} - w_{BC}^{(-)}
    + \frac{s_{BP}\chi_{BP}}{\rho_0} \hat{\rho}_P \\
  w_C &=\;
    i \left( w_+ - w_{AC}^{(-)} + w_{BC}^{(+)} \right)
    - w_{AC}^{(+)} + w_{BC}^{(-)}
    + \frac{s_{CP}\chi_{CP}}{\rho_0} \hat{\rho}_P \\
\end{align*}
  we can rewrite all the $\exp(\int d \mathbf{r} w\hat{\rho})$ type terms as
\begin{align*}
  \prod_{j}^{n_{D}P_{A}}
  \exp \left( -\omega_A(\mathbf{r}_j) \right)
  \prod_{k}^{n_{D}P_{B}}
  \exp \left( -\omega_B(\mathbf{r}_k) \right)
  \prod_{m}^{n_{G}N_{}}
  \exp \left( -\omega_C(\mathbf{r}_m) \right)
\end{align*}
where $\omega_A, \omega_B$ and $\omega_C$ are defined as
\begin{align*}
  \omega_K(\mathbf{r}) = (h \ast w_K)(\mathbf{r})
\end{align*}
Additionally, defining the bond transition probability $\Phi$ as
\begin{align*}
  \Phi(\mathbf{r} - \mathbf{r}^\prime) =
    \left( \frac{3}{2\pi b^2} \right) ^ {d/2}
    \exp \left( \frac{-3| \mathbf{r} - \mathbf{r}^\prime |^2}{2b^2} \right),
\end{align*}
we can rewrite the canonical partition function equation as
\begin{align*}
  Z_C =& \frac{1}{n_D!n_G! \left( \lambda_T^d \right)^{n_D+n_G}}
    \frac{1}{\Omega}
    \int \hdots \int \mathcal{D} \{w\} \\
    &\times
    \exp \left(
      - \frac{\rho_0}{2\kappa} \int d \mathbf{r} w_+(\mathbf{r})^2
      + i \rho_0 \int d\mathbf{r} w_+
      + \sum_{IJ,I \ne J, I \ne P}
      - \frac{\rho_0}{\chi_{IJ}}
      \int d \mathbf{r}
      \left(
        w_{IJ}^{(+)} (\mathbf{r})^2 + w_{IJ}^{(-)} (\mathbf{r})^2
      \right)
    \right) \\
    &\times \int d \mathbf{r}^{n_D P} \int d \mathbf{r}^{n_G N}
      \left(\prod_j^{n_D} 
      \left( \prod_k^{P-1}
      \Phi(\mathbf{r}_{j,k+1} - \mathbf{r}_{j,k})\right) 
      \delta \left(
        \mathbf{r}_{j,1} - \mathbf{r}_{j,\perp} 
      \right)  \right)  
    \left(   \prod_{\ell=1}^{n_D} \prod_{m=1}^{N-1}  
      \Phi (\mathbf{r}_{\ell,m+1} - \mathbf{r}_{\ell,m})\right)
        \\ 
    & \times  
    \left( \frac{3}{2\pi b^2} \right) 
    ^ {(\frac{d}{2}) (n_D (P-1) + n_G \cdot (N-1 ))  }
    \\
    &\times
    \prod_{j}^{n_{D}P_{A}}
    \exp \left( -\omega_A(\mathbf{r}_j) \right)
    \prod_{k}^{n_{D}P_{B}}
    \exp \left( -\omega_B(\mathbf{r}_k) \right)
    \prod_{m}^{n_{G}N_{}}
    \exp \left( -\omega_C(\mathbf{r}_m) \right)
\end{align*}

\textbf{==Do not trust below here ==}

Then, we define $Q_D$ as
\begin{align*}
  Q_D = \frac{1}{V} \int d\mathbf{r} q_D (N_D, \mathbf{r})
\end{align*}
where
\begin{align*}
  q_D(j+1, \mathbf{r}) = \exp(-\omega_{X_{j+1}}(\mathbf{r}))
    \int d \mathbf{r}^\prime \Phi(\mathbf{r} - \mathbf{r}^\prime)
    q(j, \mathbf{r})  
\end{align*}
where $X_{j+1}$ is either A or B depending on type of segment $j+1$
  and $q_D(1, \mathbf{r}) = \exp(-\omega_A(\mathbf{r}))$.

  We also define $Q_G$ as
  \begin{align}
    Q_G = \frac{1}{V}  \int d \mathbf{r}{q_G} (N_G, \mathbf{r})
  \end{align}
% We also define $Q_P$ as
% \begin{align*}
%   Q_P = \frac{1}{V} \int d\mathbf{r} \exp (-\omega_P)
% \end{align*}
With these definitions, we get
\begin{align*}
  Z_C =& \frac{V^{n_D+n_G}}
              {n_D!n_G! \left( \lambda_T^d \right)^{n_D+n_G}}
    \frac{1}{\Omega}
    \left( \frac{2\pi b^2}{3} \right)
    ^ {(\frac{d}{2}) (n_D (P-1) + n_G \cdot (N-1) )  }
    \int \hdots \int \mathcal{D} \{w\} \\
    &\times \exp \left(
      - \frac{\rho_0}{2\kappa} \int d \mathbf{r} w_+(\mathbf{r})^2
      + i \rho_0 \int d\mathbf{r} w_+
      - \sum_{IJ,I \ne J, I \ne P}
      \frac{\rho_0}{\chi_{IJ}}
      \int d \mathbf{r}
      \left(
        w_{IJ}^{(+)} (\mathbf{r})^2 + w_{IJ}^{(-)} (\mathbf{r})^2
      \right)
    \right) \\
    &\times Q_D^{n_D} Q_G^{n_G}
\end{align*}
We can rewrite this as
\begin{align*}
  Z_C =& \frac{V^{n_D+n_G}}
              {n_D!n_P! \left( \lambda_T^d \right)^{n_D+n_G}}
    \frac{1}{\Omega}
    \left( \frac{2\pi b^2}{3} \right)
    ^ {(\frac{d}{2}) (n_D (P-1) + n_G \cdot (N-1) )  }
    \\
    &\int \hdots \int \mathcal{D} \{w\}
    \exp \left(
      -\mathcal{H}[\{w\}]
    \right)
\end{align*}
where
\begin{align*}
  \mathcal{H}[w_+, w_{AB}^{(\pm)},w_{BC}^{(\pm)},w_{AC}^{(\pm)}] =&
    \frac{\rho_0}{2\kappa} \int d \mathbf{r} w_+(\mathbf{r})^2
    - i \rho_0 \int d\mathbf{r} w_+
    + \sum_{IJ,I \ne J, I \ne P}
    \frac{\rho_0}{\chi_{IJ}}
    \int d \mathbf{r}
    \left(
      w_{IJ}^{(+)} (\mathbf{r})^2 + w_{IJ}^{(-)} (\mathbf{r})^2
    \right) \\
    &- n_D \log Q_D - n_G \log Q_G
\end{align*}


% \section{Canonical 1S Update Derivation}

% \subsection{$w_+$ Field}

% First let's do the $w_+$ update derivation for the Canonical Ensemble.
% \begin{align*}
%   w_+^{t+1} =
%     w_+^t - \lambda \left[
%       \frac{\delta \mathcal{H}}{\delta w_+^t}
%       + \left(  \frac{\delta \mathcal{H}}{\delta w_+^{t+1}} \right) _{lin}
%       - \left(  \frac{\delta \mathcal{H}}{\delta w_+^{t}} \right) _{lin}
%     \right]
% \end{align*}
% Taking the Fourier Transform,
% \begin{align*}
%   \hat{w}_+^{t+1} =
%     \hat{w}_+^t - \lambda \left[
%       \hat{\frac{\delta \mathcal{H}}{\delta w_+^t}}
%       + \left( \hat{ \frac{\delta \mathcal{H}}{\delta w_+^{t+1}}} \right) _{lin}
%       - \left( \hat{ \frac{\delta \mathcal{H}}{\delta w_+^{t}}} \right) _{lin}
%     \right]
% \end{align*}
% Then, after plugging in the correct expressions, we can solve for
%   $\hat{w}_+^{t+1}$ and take the inverse Fourier Transform to get $w_+^{t+1}$.
% \begin{align*}
%   \frac{\delta \mathcal{H}}{\delta w_+^t} &=
%     \frac{\rho_0}{\kappa} w_+^t
%     - i\rho_0
%     + i [ (\rho_{DA,c} \ast h)(\mathbf{r})
%           + (\rho_{DB,c} \ast h)(\mathbf{r})
%           + (\rho_{P,c} \ast \Gamma)(\mathbf{r}) ] \\
%   \hat{\frac{\delta \mathcal{H}}{\delta w_+^t}} &=
%     \frac{\rho_0}{\kappa} \hat{w}_+^t
%     - i \rho_0 \delta(\mathbf{k})
%     + i [ \hat{\rho}_{DA,c} \hat{h}
%           + \hat{\rho}_{DB,c} \hat{h}
%           + \hat{\rho}_{P,c}  \hat{\Gamma} ] \\
%   \left( \hat{\frac{\delta \mathcal{H}}{\delta w_+^t}} \right) _{lin} &=
%     \frac{\rho_0}{\kappa} \hat{w}_+^t
%     - i \hat{h}^2 \phi_D \rho_0 N_D
%       (\hat{g}_{AA} + 2 \hat{g}_{AB} + \hat{g}_{BB}) i \hat{w}_+^t
%     - i \phi_P \rho_0 \hat{\Gamma}^2 i \hat{w}_+^t \\
%   &= \frac{\rho_0}{\kappa} \hat{w}_+^t
%     + \hat{h}^2 \phi_D \rho_0 N_D
%       (\hat{g}_{AA} + 2 \hat{g}_{AB} + \hat{g}_{BB}) \hat{w}_+^t
%     + \phi_P \rho_0 \hat{\Gamma}^2 \hat{w}_+^t
% \end{align*}
% Assembling the pieces, we get
% \begin{align*}
%   \hat{w}_+^{t+1} =
%     \hat{w}_+^t - \lambda \left[
%       \frac{\rho_0}{\kappa} \hat{w}_+^t
%       - i \rho_0 \delta(\mathbf{k})
%       + i ( \hat{\rho}_{DA,c} \hat{h}
%             + \hat{\rho}_{DB,c} \hat{h}
%             + \hat{\rho}_{P,c}  \hat{\Gamma} )
%       + A (\hat{w}_+^{t+1} - \hat{w}_+^t)
%     \right]
% \end{align*}
% where
% \begin{align*}
%   A &=
%   \frac{1}{w_+^t}
%   \left( \hat{\frac{\delta \mathcal{H}}{\delta w_+^t}} \right) _{lin}
%   =
%   \frac{1}{w_+^{t+1}}
%   \left( \hat{\frac{\delta \mathcal{H}}{\delta w_+^{t+1}}} \right) _{lin} \\
%   &=
%   \frac{\rho_0}{\kappa}
%       + \hat{h}^2 \phi_D \rho_0 N_D
%         (\hat{g}_{AA} + 2 \hat{g}_{AB} + \hat{g}_{BB})
%       + \phi_P \rho_0 \hat{\Gamma}^2
% \end{align*}
% If we also let $B$ and $F$ equal
% \begin{align*}
%   B &= A - \frac{\rho_0}{\kappa} \\
%     &= \hat{h}^2 \phi_D \rho_0 N_D
%         (\hat{g}_{AA} + 2 \hat{g}_{AB} + \hat{g}_{BB})
%       + \phi_P \rho_0 \hat{\Gamma}^2 \\
%   F &= - i \rho_0 \delta (\mathbf{k})
%        + i ( \hat{\rho}_{DA,c} \hat{h}
%             + \hat{\rho}_{DB,c} \hat{h}
%             + \hat{\rho}_{P,c}  \hat{\Gamma} )
% \end{align*}
% Then
% \begin{align*}
%   \hat{w}_+^{t+1} ( 1 + \lambda A ) =
%   \hat{w}_+^t - \lambda \left( F - B \hat{w}_+^t \right) \\
%   \hat{w}_+^{t+1} =
%   \frac{\hat{w}_+^t - \lambda \left( F - B \hat{w}_+^t \right)}
%        {1 + \lambda A}
% \end{align*}

% \subsection{$w_{AB}^{(+)}$}

% For the $w_{AB}^{(+)}$ field, the relevant expressions are:
% \begin{align*}
%   \left. \frac{\delta \mathcal{H}}{\delta  w_{AB}^{(+)} } \right|_t &=
%     \frac{2\rho_0}{\chi_{AB}} \left. w_{AB}^{(+)} \right|_t
%     + i [ (\rho_{DA,c} \ast h)(\mathbf{r})
%           + (\rho_{DB,c} \ast h)(\mathbf{r}) ] \\
%     \left. \hat{\frac{\delta \mathcal{H}}{\delta w_{AB}^{(+)}}} \right|_t &=
%     \frac{2\rho_0}{\chi_{AB}} \left. \hat{w}_{AB}^{(+)} \right|_t
%     + i [ \hat{\rho}_{DA,c} \hat{h}
%           + \hat{\rho}_{DB,c} \hat{h} ] \\
%   \left. \hat{\frac{\delta \mathcal{H}}{\delta w_{AB}^{(+)}}} \right| ^{lin}_t &=
%     \frac{2\rho_0}{\chi_{AB}} \left. \hat{w}_{AB}^{(+)} \right|_t
%     - i \hat{h}^2 \phi_D \rho_0 N_D
%       (\hat{g}_{AA} + 2 \hat{g}_{AB} + \hat{g}_{BB}) i \left.
%         \hat{w}_{AB}^{(+)} \right|_t \\
%   &= \frac{2\rho_0}{\chi_{AB}} \left. \hat{w}_{AB}^{(+)} \right|_t
%     + \hat{h}^2 \phi_D \rho_0 N_D
%       (\hat{g}_{AA} + 2 \hat{g}_{AB} + \hat{g}_{BB})
%         \left. \hat{w}_{AB}^{(+)} \right|_t
% \end{align*}
% This gives us
% \begin{align*}
%   \left. \hat{w}_{AB}^{(+)} \right|_{t+1} =
%     \left. \hat{w}_{AB}^{(+)} \right|_t - \lambda \left[
%       \frac{2\rho_0}{\chi_{AB}} \left. \hat{w}_{AB}^{(+)} \right|_t
%       + i ( \hat{\rho}_{DA,c} \hat{h}
%             + \hat{\rho}_{DB,c} \hat{h} )
%       + A ( \left. \hat{w}_{AB}^{(+)} \right|_{t+1}
%             - \left. \hat{w}_{AB}^{(+)} \right|_t)
%     \right]
% \end{align*}
% where
% \begin{align*}
%   A &=
%   \frac{2\rho_0}{\chi_{AB}}
%       + \hat{h}^2 \phi_D \rho_0 N_D
%         (\hat{g}_{AA} + 2 \hat{g}_{AB} + \hat{g}_{BB})
% \end{align*}
% If we let $B$ and $F$ equal
% \begin{align*}
%   B &= A - \frac{2\rho_0}{\chi_{AB}} \\
%     &= \hat{h}^2 \phi_D \rho_0 N_D
%         (\hat{g}_{AA} + 2 \hat{g}_{AB} + \hat{g}_{BB}) \\
%   F &= + i ( \hat{\rho}_{DA,c} \hat{h}
%             + \hat{\rho}_{DB,c} \hat{h} )
% \end{align*}
% Then
% \begin{align*}
%   \left. \hat{w}_{AB}^{(+)} \right|_{t+1} ( 1 + \lambda A ) =
%     \left. \hat{w}_{AB}^{(+)} \right|_t
%     - \lambda \left( F - B \left. \hat{w}_{AB}^{(+)} \right|_t \right) \\
%   \left. \hat{w}_{AB}^{(+)} \right|_{t+1} =
%   \frac{\left. \hat{w}_{AB}^{(+)} \right|_t - \lambda
%           \left( F - B \left. \hat{w}_{AB}^{(+)} \right|_t \right)}
%        {1 + \lambda A}
% \end{align*}

% \subsection{$w_{AB}^{(-)}$ Field}

% For the $w_{AB}^{(-)}$ field, the relevant expressions are:
% \begin{align*}
%   \left. \frac{\delta \mathcal{H}}{\delta w_{AB}^{(-)}} \right|_t &=
%     \frac{2 \rho_0}{\chi_{AB}} \left. w_{AB}^{(-)} \right|_t
%     - (\rho_{DA,c} \ast h)(\mathbf{r})
%     + (\rho_{DB,c} \ast h)(\mathbf{r}) \\
%   \left. \hat{\frac{\delta \mathcal{H}}{\delta w_{AB}^{(-)}}} \right|_t &=
%     \frac{2 \rho_0}{\chi_{AB}} \left. \hat{w}_{AB}^{(-)} \right|_t
%     - \hat{\rho}_{DA,c} \hat{h}
%     + \hat{\rho}_{DB,c} \hat{h} \\
%   \left.
%     \hat{\frac{\delta \mathcal{H}}{\delta w_{AB}^{(-)}}}
%   \right| ^{lin}_t &=
%     \frac{2\rho_0}{\chi_{AB}} \left. \hat{w}_{AB}^{(-)} \right|_t
% \end{align*}
% Note that we don't do the weak inhomogeneity expansion here because the
%   $w_{AB}^{(-)}$ field tends to be much less stiff than the $w_+$ fields and so
%   doesn't need the extra approximation.
% Now we get
% \begin{align*}
%   \left. \hat{w}_{AB}^{(-)} \right|_{t+1} &=
%     \left. \hat{w}_{AB}^{(-)} \right|_t - \lambda \left[
%       \frac{2\rho_0}{\chi_{AB}} \left. \hat{w}_{AB}^{(-)} \right|_t
%       - \hat{\rho}_{DA,c} \hat{h}
%       + \hat{\rho}_{DB,c} \hat{h}
%       + \frac{2\rho_0}{\chi_{AB}}
%         ( \left. \hat{w}_{AB}^{(-)}\right|_{t+1}
%           - \left. \hat{w}_{AB}^{(-)} \right|_t
%         )
%     \right] \\
%   \left. \hat{w}_{AB}^{(-)} \right|_{t+1} ( 1 + \lambda \frac{2
%     \rho_0}{\chi_{AB}} ) &=
%     \left. \hat{w}_{AB}^{(-)} \right|_t - \lambda \left(
%       - \hat{\rho}_{DA,c} \hat{h}
%       + \hat{\rho}_{DB,c} \hat{h}
%     \right) \\
%   \left. \hat{w}_{AB}^{(-)} \right|_{t+1} &=
%     \frac{
%       \left. \hat{w}_{AB}^{(-)} \right|_t - \lambda \left(
%         - \hat{\rho}_{DA,c} \hat{h}
%         + \hat{\rho}_{DB,c} \hat{h}
%       \right)
%     }
%     {
%       \left( 1 + \lambda \frac{2 \rho_0}{\chi_{AB}} \right)
%     }
% \end{align*}

% \subsection{$w_{AC}^{(-)}$}

% For the $w_{AC}^{(-)}$ field, the relevant expressions are:
% \begin{align*}
%   \left. \frac{\delta \mathcal{H}}{\delta  w_{AC}^{(-)} } \right|_t &=
%     \frac{2\rho_0}{\chi_{AC}} \left. w_{AC}^{(-)} \right|_t
%     + i [ (\rho_{DA,c} \ast h)(\mathbf{r})
%           + (\rho_{P,c} \ast \Gamma)(\mathbf{r}) ] \\
%     \left. \hat{\frac{\delta \mathcal{H}}{\delta w_{AC}^{(-)}}} \right|_t &=
%     \frac{2\rho_0}{\chi_{AC}} \left. \hat{w}_{AC}^{(-)} \right|_t
%     + i [ \hat{\rho}_{DA,c} \hat{h}
%           + \hat{\rho}_{P,c}  \hat{\Gamma} ] \\
%   \left. \hat{\frac{\delta \mathcal{H}}{\delta w_{AC}^{(-)}}} \right| ^{lin}_t &=
%     \frac{2\rho_0}{\chi_{AC}} \left. \hat{w}_{AC}^{(-)} \right|_t
%     - i \hat{h}^2 f_A \phi_D \rho_0 N_D
%       (\hat{g}_{AA} + 2 \hat{g}_{AB} + \hat{g}_{BB}) i \left.
%         \hat{w}_{AC}^{(-)} \right|_t
%     - i \phi_P \rho_0 \hat{\Gamma}^2 i \left. \hat{w}_{AC}^{(-)} \right|_t \\
%   &= \frac{2\rho_0}{\chi_{AC}} \left. \hat{w}_{AB}^{(-)} \right|_t
%     + \hat{h}^2 f_A \phi_D \rho_0 N_D
%       (\hat{g}_{AA} + 2 \hat{g}_{AB} + \hat{g}_{BB})
%         \left. \hat{w}_{AC}^{(-)} \right|_t
%     + \phi_P \rho_0 \hat{\Gamma}^2 \left. \hat{w}_{AC}^{(-)} \right|_t
% \end{align*}
% This gives us
% \begin{align*}
%   \left. \hat{w}_{AC}^{(-)} \right|_{t+1} =
%     \left. \hat{w}_{AC}^{(-)} \right|_t - \lambda \left[
%       \frac{2\rho_0}{\chi_{AC}} \left. \hat{w}_{AC}^{(-)} \right|_t
%       + i ( \hat{\rho}_{DA,c} \hat{h}
%             + \hat{\rho}_{P,c}  \hat{\Gamma} )
%       + A ( \left. \hat{w}_{AC}^{(-)} \right|_{t+1}
%             - \left. \hat{w}_{AC}^{(-)} \right|_t)
%     \right]
% \end{align*}
% where
% \begin{align*}
%   A &=
%   \frac{2\rho_0}{\chi_{AC}}
%       + \hat{h}^2 f_A \phi_D \rho_0 N_D
%         (\hat{g}_{AA} + 2 \hat{g}_{AB} + \hat{g}_{BB})
%       + \phi_P \rho_0 \hat{\Gamma}^2
% \end{align*}
% If we let $B$ and $F$ equal
% \begin{align*}
%   B &= A - \frac{2\rho_0}{\chi_{AC}} \\
%     &= \hat{h}^2 f_A \phi_D \rho_0 N_D
%         (\hat{g}_{AA} + 2 \hat{g}_{AB} + \hat{g}_{BB})
%       + \phi_P \rho_0 \hat{\Gamma}^2 \\
%   F &= + i ( \hat{\rho}_{DA,c} \hat{h}
%             + \hat{\rho}_{P,c}  \hat{\Gamma} )
% \end{align*}
% Then
% \begin{align*}
%   \left. \hat{w}_{AC}^{(-)} \right|_{t+1} ( 1 + \lambda A ) =
%     \left. \hat{w}_{AC}^{(-)} \right|_t
%     - \lambda \left( F - B \left. \hat{w}_{AC}^{(-)} \right|_t \right) \\
%   \left. \hat{w}_{AC}^{(-)} \right|_{t+1} =
%   \frac{\left. \hat{w}_{AC}^{(-)} \right|_t - \lambda
%           \left( F - B \left. \hat{w}_{AC}^{(-)} \right|_t \right)}
%        {1 + \lambda A}
% \end{align*}

% \subsection{$w_{AC}^{(+)}$ Field}

% For the $w_{AC}^{(+)}$ field, the relevant expressions are:
% \begin{align*}
%   \left. \frac{\delta \mathcal{H}}{\delta w_{AC}^{(+)}} \right|_t &=
%     \frac{2 \rho_0}{\chi_{AC}} \left. w_{AC}^{(+)} \right|_t
%     - (\rho_{DA,c} \ast h)(\mathbf{r})
%     - (\rho_{P,c} \ast \Gamma)(\mathbf{r}) \\
%   \left. \hat{\frac{\delta \mathcal{H}}{\delta w_{AC}^{(+)}}} \right|_t &=
%     \frac{2 \rho_0}{\chi_{AC}} \left. \hat{w}_{AC}^{(+)} \right|_t
%     - \hat{\rho}_{DA,c} \hat{h}
%     - \hat{\rho}_{P,c}  \hat{\Gamma} \\
%   \left.
%     \hat{\frac{\delta \mathcal{H}}{\delta w_{AC}^{(+)}}}
%   \right| ^{lin}_t &=
%     \frac{2\rho_0}{\chi_{AC}} \left. \hat{w}_{AC}^{(+)} \right|_t
% \end{align*}
% Note that we don't do the weak inhomogeneity expansion here because the
%   $w_{AC}^{(+)}$ field tends to be much less stiff than the $w_+$ fields and so
%   doesn't need the extra approximation.
% Now we get
% \begin{align*}
%   \left. \hat{w}_{AC}^{(+)} \right|_{t+1} &=
%     \left. \hat{w}_{AC}^{(+)} \right|_t - \lambda \left[
%       \frac{2\rho_0}{\chi_{AC}} \left. \hat{w}_{AC}^{(+)} \right|_t
%       - \hat{\rho}_{DA,c} \hat{h}
%       - \hat{\rho}_{P,c}  \hat{\Gamma}
%       + \frac{2\rho_0}{\chi_{AC}}
%         ( \left. \hat{w}_{AC}^{(+)}\right|_{t+1}
%           - \left. \hat{w}_{AC}^{(+)} \right|_t
%         )
%     \right] \\
%   \left. \hat{w}_{AC}^{(+)} \right|_{t+1} ( 1 + \lambda \frac{2
%     \rho_0}{\chi_{AC}} ) &=
%     \left. \hat{w}_{AC}^{(+)} \right|_t - \lambda \left(
%       - \hat{\rho}_{DA,c} \hat{h}
%       - \hat{\rho}_{P,c}  \hat{\Gamma}
%     \right) \\
%   \left. \hat{w}_{AC}^{(+)} \right|_{t+1} &=
%     \frac{
%       \left. \hat{w}_{AC}^{(+)} \right|_t - \lambda \left(
%         - \hat{\rho}_{DA,c} \hat{h}
%         - \hat{\rho}_{P,c}  \hat{\Gamma}
%       \right)
%     }
%     {
%       \left( 1 + \lambda \frac{2 \rho_0}{\chi_{AC}} \right)
%     }
% \end{align*}

% \subsection{$w_{BC}^{(+)}$}

% For the $w_{BC}^{(+)}$ field, the relevant expressions are:
% \begin{align*}
%   \left. \frac{\delta \mathcal{H}}{\delta  w_{AC}^{(-)} } \right|_t &=
%     \frac{2\rho_0}{\chi_{AC}} \left. w_{AC}^{(-)} \right|_t
%     + i [ (\rho_{DA,c} \ast h)(\mathbf{r})
%           + (\rho_{P,c} \ast \Gamma)(\mathbf{r}) ] \\
%     \left. \hat{\frac{\delta \mathcal{H}}{\delta w_{AC}^{(-)}}} \right|_t &=
%     \frac{2\rho_0}{\chi_{AC}} \left. \hat{w}_{AC}^{(-)} \right|_t
%     + i [ \hat{\rho}_{DA,c} \hat{h}
%           + \hat{\rho}_{P,c}  \hat{\Gamma} ] \\
%   \left. \hat{\frac{\delta \mathcal{H}}{\delta w_{AC}^{(-)}}} \right| ^{lin}_t &=
%     \frac{2\rho_0}{\chi_{AC}} \left. \hat{w}_{AC}^{(-)} \right|_t
%     - i \hat{h}^2 f_A \phi_D \rho_0 N_D
%       (\hat{g}_{AA} + 2 \hat{g}_{AB} + \hat{g}_{BB}) i \left.
%         \hat{w}_{AC}^{(-)} \right|_t
%     - i \phi_P \rho_0 \hat{\Gamma}^2 i \left. \hat{w}_{AC}^{(-)} \right|_t \\
%   &= \frac{2\rho_0}{\chi_{AC}} \left. \hat{w}_{AB}^{(-)} \right|_t
%     + \hat{h}^2 f_A \phi_D \rho_0 N_D
%       (\hat{g}_{AA} + 2 \hat{g}_{AB} + \hat{g}_{BB})
%         \left. \hat{w}_{AC}^{(-)} \right|_t
%     + \phi_P \rho_0 \hat{\Gamma}^2 \left. \hat{w}_{AC}^{(-)} \right|_t
% \end{align*}
% This gives us
% \begin{align*}
%   \left. \hat{w}_{AC}^{(-)} \right|_{t+1} =
%     \left. \hat{w}_{AC}^{(-)} \right|_t - \lambda \left[
%       \frac{2\rho_0}{\chi_{AC}} \left. \hat{w}_{AC}^{(-)} \right|_t
%       + i ( \hat{\rho}_{DA,c} \hat{h}
%             + \hat{\rho}_{P,c}  \hat{\Gamma} )
%       + A ( \left. \hat{w}_{AC}^{(-)} \right|_{t+1}
%             - \left. \hat{w}_{AC}^{(-)} \right|_t)
%     \right]
% \end{align*}
% where
% \begin{align*}
%   A &=
%   \frac{2\rho_0}{\chi_{AC}}
%       + \hat{h}^2 f_A \phi_D \rho_0 N_D
%         (\hat{g}_{AA} + 2 \hat{g}_{AB} + \hat{g}_{BB})
%       + \phi_P \rho_0 \hat{\Gamma}^2
% \end{align*}
% If we let $B$ and $F$ equal
% \begin{align*}
%   B &= A - \frac{2\rho_0}{\chi_{AC}} \\
%     &= \hat{h}^2 f_A \phi_D \rho_0 N_D
%         (\hat{g}_{AA} + 2 \hat{g}_{AB} + \hat{g}_{BB})
%       + \phi_P \rho_0 \hat{\Gamma}^2 \\
%   F &= + i ( \hat{\rho}_{DA,c} \hat{h}
%             + \hat{\rho}_{P,c}  \hat{\Gamma} )
% \end{align*}
% Then
% \begin{align*}
%   \left. \hat{w}_{AC}^{(-)} \right|_{t+1} ( 1 + \lambda A ) =
%     \left. \hat{w}_{AC}^{(-)} \right|_t
%     - \lambda \left( F - B \left. \hat{w}_{AC}^{(-)} \right|_t \right) \\
%   \left. \hat{w}_{AC}^{(-)} \right|_{t+1} =
%   \frac{\left. \hat{w}_{AC}^{(-)} \right|_t - \lambda
%           \left( F - B \left. \hat{w}_{AC}^{(-)} \right|_t \right)}
%        {1 + \lambda A}
% \end{align*}

% \subsection{$w_{BC}^{(-)}$ Field}

% For the $w_{BC}^{(-)}$ field, the relevant expressions are:
% \begin{align*}
%   \left. \frac{\delta \mathcal{H}}{\delta w_{AC}^{(+)}} \right|_t &=
%     \frac{2 \rho_0}{\chi_{AC}} \left. w_{AC}^{(+)} \right|_t
%     - (\rho_{DA,c} \ast h)(\mathbf{r})
%     - (\rho_{P,c} \ast \Gamma)(\mathbf{r}) \\
%   \left. \hat{\frac{\delta \mathcal{H}}{\delta w_{AC}^{(+)}}} \right|_t &=
%     \frac{2 \rho_0}{\chi_{AC}} \left. \hat{w}_{AC}^{(+)} \right|_t
%     - \hat{\rho}_{DA,c} \hat{h}
%     - \hat{\rho}_{P,c}  \hat{\Gamma} \\
%   \left.
%     \hat{\frac{\delta \mathcal{H}}{\delta w_{AC}^{(+)}}}
%   \right| ^{lin}_t &=
%     \frac{2\rho_0}{\chi_{AC}} \left. \hat{w}_{AC}^{(+)} \right|_t
% \end{align*}
% Note that we don't do the weak inhomogeneity expansion here because the
%   $w_{AC}^{(+)}$ field tends to be much less stiff than the $w_+$ fields and so
%   doesn't need the extra approximation.
% Now we get
% \begin{align*}
%   \left. \hat{w}_{AC}^{(+)} \right|_{t+1} &=
%     \left. \hat{w}_{AC}^{(+)} \right|_t - \lambda \left[
%       \frac{2\rho_0}{\chi_{AC}} \left. \hat{w}_{AC}^{(+)} \right|_t
%       - \hat{\rho}_{DA,c} \hat{h}
%       - \hat{\rho}_{P,c}  \hat{\Gamma}
%       + \frac{2\rho_0}{\chi_{AC}}
%         ( \left. \hat{w}_{AC}^{(+)}\right|_{t+1}
%           - \left. \hat{w}_{AC}^{(+)} \right|_t
%         )
%     \right] \\
%   \left. \hat{w}_{AC}^{(+)} \right|_{t+1} ( 1 + \lambda \frac{2
%     \rho_0}{\chi_{AC}} ) &=
%     \left. \hat{w}_{AC}^{(+)} \right|_t - \lambda \left(
%       - \hat{\rho}_{DA,c} \hat{h}
%       - \hat{\rho}_{P,c}  \hat{\Gamma}
%     \right) \\
%   \left. \hat{w}_{AC}^{(+)} \right|_{t+1} &=
%     \frac{
%       \left. \hat{w}_{AC}^{(+)} \right|_t - \lambda \left(
%         - \hat{\rho}_{DA,c} \hat{h}
%         - \hat{\rho}_{P,c}  \hat{\Gamma}
%       \right)
%     }
%     {
%       \left( 1 + \lambda \frac{2 \rho_0}{\chi_{AC}} \right)
%     }
% \end{align*}

% \section{Calculating Densities}

% \subsection{Canonical Ensemble}

% In the Canonical Ensemble, the polymer densities are given by
% \begin{align*}
%   \rho_{DA,c} &=
%     -n_D \frac{\delta \log Q_D}{\delta \omega_A(\mathbf{r})}
%     = \frac{n_D}{V Q_D}
%     \sum_{j=1}^{P_A}
%     q_D(j, \mathbf{r})
%     e^{\omega_A(\mathbf{r})}
%     q_D^\dagger(P-j, \mathbf{r}) \\
%   \rho_{DB,c} &=
%     -n_D \frac{\delta \log Q_D}{\delta \omega_B(\mathbf{r})}
%     = \frac{n_D}{V Q_D}
%     \sum_{j=P_A+1}^{P}
%     q_D(j, \mathbf{r})
%     e^ {\omega_B(\mathbf{r})}
%     q_D^\dagger(P-j, \mathbf{r})
% \end{align*}
% and the particle density is given by
% \begin{align*}
%   \rho_P(\mathbf{r}) =
%     -n_P \frac{\delta \log Q_P}{\delta \omega_P(\mathbf{r})}
%     = \frac{n_P}{V Q_P} e^{-\omega_P(\mathbf{r})}
% \end{align*}

% \subsection{Grand Canonical Ensemble}

% In the Grand Canonical Ensemble, the polymer densities are given by
% \begin{align*}
%   \rho_{DA,c} &=
%     -z_D V \frac{\delta Q_D}{\delta \omega_A(\mathbf{r})}
%     = z_D
%     \sum_{j=1}^{P_A}
%     q_D(j, \mathbf{r})
%     e^{\omega_A(\mathbf{r})}
%     q_D^\dagger(P-j, \mathbf{r}) \\
%   \rho_{DB,c} &=
%     -z_D \frac{\delta Q_D}{\delta \omega_B(\mathbf{r})}
%     = z_D
%     \sum_{j=P_A+1}^{P}
%     q_D(j, \mathbf{r})
%     e^ {\omega_B(\mathbf{r})}
%     q_D^\dagger(P-j, \mathbf{r})
% \end{align*}
% and the particle density is given by
% \begin{align*}
%   \rho_P(\mathbf{r}) =
%     -z_P \frac{\delta Q_P}{\delta \omega_P(\mathbf{r})}
%     = z_P e^{-\omega_P(\mathbf{r})}
% \end{align*}

% \section{Comparing $z_D$ values with 2 Field Model}

% From the 2-field model derivation,
% \begin{align*}
%   z_{D0,2} = \frac{1}{\lambda_T^d}
%   \exp \left( -\frac{N_D\chi}{4}  \right)
%   \left( \frac{2\pi b^2}{3} \right)^{d/2(N_D-1)} 
% \end{align*}
% And from the 3-field model derivation, (this document)
% \begin{align*}
%   z_{D0,3} = \frac{1}{\lambda_T^d}
%   \left( \frac{2\pi b^2}{3} \right)^{d/2(N_D-1)} 
% \end{align*}
% Therefore, to compare 2-field and 3-field simulations, we need to take this
%   into account to make sure $\mu_D$ matches between them.
% Thus, given a 3-field model using $z_{D,3}$, the corresponsing value of
%   $z_{D,2}$ necessary to match a 2-field model is
% \begin{align*}
%   z_{D,2} = \exp \left( -\frac{N_D\chi}{4}  \right) z_{D,3}
% \end{align*}

% \section{Simplification for Homogeneous System}

% In a homogeneous system, the fields $w_+$, $w_{AB}^{(+)}$, and $w_{AB}^{(-)}$
%   are all constants.
% To simplify solving the equations, it is convenient to use only real numbers by
%   solving for $iw_+$, $iw_{AB}^{(+)}$, and $w_{AB}^{(-)}$.
% With this in mind, we can rewrite the Grand Canonical Hamiltonian as

% \begin{align*}
%   H_G = - &\frac{\rho_0 V}{2\kappa} (i w_+)^2
%         - \rho_0 V (i w_+)
%         - \frac{\rho_0 V}{\chi} (i w_{AB}^{(+)})^2
%         + \frac{\rho_0 V}{\chi} (w_{AB}^{(-)})^2
%         - z_D V Q_D
%         - z_P V Q_P
% \end{align*}

% If we solve with simple Euler equations, we get

% \begin{align*}
%   iw_+^{t+1} &= iw_+^t - i\lambda_+ \frac{\partial H_G/V}{\partial w_+^t} \\
%     &= iw_+^t + \lambda_+ \frac{\partial H_G/V}{\partial iw_+^t} \\
%   \left. iw_{AB}^{(+)} \right|_{t+1} &= \left. iw_{AB}^{(+)} \right|_t
%     - i\lambda_+ \left. \frac{\partial H_G/V}{\partial w_{AB}^{(+)}} \right|_t \\
%     &= \left. iw_{AB}^{(+)} \right|_t
%     + \lambda_+ \left. \frac{\partial H_G/V}{\partial iw_{AB}^{(+)}} \right|_t \\
%   \left. w_{AB}^{(-)} \right|_{t+1} &= \left. w_{AB}^{(-)} \right|_t
%     - \lambda_- \left. \frac{\partial H_G/V}{\partial w_{AB}^{(-)}} \right|_t
% \end{align*}

% With this in mind, we can rewrite the partition functions as

% \begin{align*}
%   Q_D &= \exp(-P_A w_A - P_B w_B) \\
%   Q_P &= \exp(-\rho_0 V_P w_A)
% \end{align*}

% Let's note that

% \begin{align*}
%   \rho_D &= \rho_{DA} + \rho_{DB} \\
%   &= -z_D \frac{\partial Q_D}{\partial \omega_A}
%      -z_D \frac{\partial Q_D}{\partial \omega_B} \\
%   &= -z_D \frac{\partial Q_D}{\partial w_A}
%      -z_D \frac{\partial Q_D}{\partial w_B} \\
%   &= z_D P \exp(-P_A w_A - P_B w_B) \\
%   &= z_D P Q_D
% \end{align*}

% and

% \begin{align*}
%   \rho_{P,c} &= -z_P \frac{\partial Q_P}{\partial \omega_P} \\
%   &= z_P \exp(-\rho_0 V_P w_A) \\
%   \rho_{P,c} &= z_P Q_P \\
%   \rho_P &= \rho_0 V_P z_P Q_P
% \end{align*}

% with the following derivatives:

% \begin{align*}
%   \frac{\partial H_G/V}{\partial iw_+} &=
%   - \frac{\rho_0}{\kappa} (iw_+)
%   - \rho_0
%   - \frac{2 \rho_0}{\chi} (iw_{AB}^{(+)})
%   + \frac{2 \rho_0}{\chi} (w_{AB}^{(-)})
%   - z_D \frac{\partial Q_D}{\partial iw_+}
%   - z_P \frac{\partial Q_P}{\partial iw_+}
% \end{align*}

% With this in mind, we can rewrite the partition functions as

% \begin{align*}
%   Q_D &= \exp(-f P w_A - (1-f) P w_B) \\
%   &= \exp(-f P (iw_+ + iw_{AB}^{(+)} - w_{AB}^{(-)})
%           - (1-f) P (iw_+ + iw_{AB}^{(+)} + w_{AB}^{(-)})
%          ) \\
%   &= \exp(-P(iw_+ + iw_{AB}^{(+)}) + (2f-1)P w_{AB}^{(-)})
% \end{align*}

% and \begin{align*}
%   Q_P &= \exp(-\rho_0 V_P w_A) \\
%       &= \exp(-\rho_0 V_P (iw_+ + iw_{AB}^{(+)} - w_{AB}^{(-)}))
% \end{align*}

% Then we can rewrite the Grand Canonical Hamiltonian as

% \begin{align*}
%   H_G = - &\frac{\rho_0 V}{2\kappa} (i w_+)^2
%         - \rho_0 V (i w_+)
%         - \frac{\rho_0 V}{\chi} (i w_{AB}^{(+)})^2
%         + \frac{\rho_0 V}{\chi} (w_{AB}^{(-)})^2 \\
%         - &z_D V \exp(-P(iw_+ + iw_{AB}^{(+)}) + (2f-1)P w_{AB}^{(-)}) \\
%         - &z_P V \exp(-\rho_0 V_P (iw_+ + iw_{AB}^{(+)} - w_{AB}^{(-)}))
% \end{align*}

% If we solve with simple Euler equations, we get

% \begin{align*}
%   w_+^{t+1} &= w_+^t - \lambda_+ \frac{\partial H_G/V}{\partial w_+^t} \\
%   \left. w_{AB}^{(+)} \right|_{t+1} &= \left. w_{AB}^{(+)} \right|_t
%     - \lambda_+ \left. \frac{\partial H_G/V}{\partial w_{AB}^{(+)}} \right|_t \\
%   \left. w_{AB}^{(-)} \right|_{t+1} &= \left. w_{AB}^{(-)} \right|_t
%     - \lambda_- \left. \frac{\partial H_G/V}{\partial w_{AB}^{(-)}} \right|_t
% \end{align*}

% with the following derivatives:

% \begin{align*}
%   \frac{\partial H}{\partial w_+^t} &=
%   - \frac{\rho_0}{\kappa} (iw_+)
%   - \rho_0
%   - \frac{2 \rho_0}{\chi} (iw_{AB}^{(+)})
%   + \frac{2 \rho_0}{\chi} (w_{AB}^{(-)})
% \end{align*}

\end{document}
