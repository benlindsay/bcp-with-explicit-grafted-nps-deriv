\documentclass{article}

\usepackage[utf8]{inputenc}
\usepackage[margin=1.0in]{geometry}
\usepackage{color}
\usepackage{amsmath}

\begin{document}

\begin{center}
  \textbf{Weakly Compressible AB Diblock + Bare A-like Nanoparticle Derivation
    with Smearing}
\end{center}

\section{Canonical Ensemble Derivation (with $w_+$ and $w_{AB}^{(\pm)}$ fields)}

The system is composed of $n_D$ AB diblock chains and $n_P$ bare, A-like
  nanoparticles.
Each diblock chain has $P_A + P_B = P$ segments.
$\chi$ is the interaction strength between A and B components.
Segment center densities are defined as
\begin{align*}
  \hat{\rho}_{DA,c} (\mathbf{r}) =&
    \sum_{i=1}^{n_D} \sum_{j=1}^{P_A}
    \delta(\mathbf{r} - \mathbf{r}_{i,j}) \\
  \hat{\rho}_{DB,c} (\mathbf{r}) =&
    \sum_{i=1}^{n_D} \sum_{j=P_A+1}^{P}
    \delta(\mathbf{r} - \mathbf{r}_{i,j}) \\
  \hat{\rho}_{P,c} (\mathbf{r}) =&
    \sum_{i=1}^{n_P}
    \delta(\mathbf{r} - \mathbf{r}_i)
\end{align*}
For all of the polymer segments, the full (smeared) segment densities are given
  by:
\begin{align*}
  \hat{\rho}_K(\mathbf{r}) = (h \ast \hat{\rho}_{K,c})(\mathbf{r})
\end{align*}
where $K \in \{ DA, DB\}$ and $h$ is the segment density distribution
  function given by the Gaussian
\begin{align*}
  h(\mathbf{r}) = \left( \frac{1}{2\pi a^2} \right)^{d/2}
  \exp \left( - \frac{|\mathbf{r}|^2}{2a^2}  \right)
\end{align*}
where $a$ is the segment size and $d$ is the number of dimensions.
The full nanoparticle density distribution is given by
\begin{align*}
  \hat{\rho}_P = (\Gamma \ast \hat{\rho}_{P,c}) (\mathbf{r})
\end{align*}
where
\begin{align*}
  \Gamma(\mathbf{r}) = \frac{\rho_0}{2}
    \textrm{erfc} \left( \frac{|\mathbf{r}| - R_P}{\xi}  \right)
\end{align*}
where $R_P$ is the nanoparticle radius, $\rho_0$ is the bulk density,
  and $\xi$ controls the nanoparticle interface width.

The harmonic bond potential between connected segments is given by
\begin{align*}
  \beta U_0 =   \sum_{i=1}^{n_D}    \sum_{j=1}^{P-1}
    \frac{3 \left| \mathbf{r}_{i,j+1} - \mathbf{r}_{i,j} \right| ^ 2 }
         { 2 b^2 }
\end{align*}
The nonbonded interaction potential is given by
\begin{align*}
  \beta U_1 = \frac{\chi}{\rho_0} \int d\mathbf{r} \hat{\rho}_A \hat{\rho}_B
\end{align*}
where $\hat{\rho}_A = \hat{\rho}_{DA}$ if the particles are neutral or
  $\hat{\rho}_A = \hat{\rho}_{DA} + \hat{\rho}_P$ if the particles are A-like,
  and $\hat{\rho}_B = \hat{\rho}_{DB}$.
A Helfand incompressibility potential penalizes deviations away from $\rho_0$,
  and is given by
\begin{align*}
  \beta U_2 = \frac{\kappa}{2 \rho_0} \int d \mathbf{r}
    \left[ \hat{\rho}_+ (\mathbf{r}) - \rho_0 \right] ^ 2
\end{align*}
where $\hat{\rho}_+ = \hat{\rho}_{DA} + \hat{\rho}_{DB} + \hat{\rho}_P$
  is the local total density.

This gives us a canonical partition function of
\begin{align*}
  Z_C = \frac{1}{n_D!n_P! \left( \lambda_T^3 \right)^{n_D+n_P}}
    \int d \mathbf{r}^{n_DN_D} \int d \mathbf{r}^{n_P}
    \exp \left( -\beta U_0 - \beta U_1 - \beta U_2 \right)
\end{align*}
To prepare this for a particle-to-field transformation, let's define
\begin{align*}
  \hat{\rho}_{\pm} (\mathbf{r}) =
    \hat{\rho}_A(\mathbf{r}) \pm \hat{\rho}_B(\mathbf{r})
\end{align*}
With this definition, we can rewrite $\beta U_1$ as
\begin{align*}
  \beta U_1 = \frac{\chi}{4\rho_0}
    \int d \mathbf{r}
    \left( \hat{\rho}_+(\mathbf{r})^2 - \hat{\rho}_-(\mathbf{r})^2 \right)
\end{align*}
From there, using the Gaussian functional integral, we get
\begin{align*}
  \exp(-\beta U_1) =&
    \frac{1}{\Omega_{AB}^{(+)} \Omega_{AB}^{(-)}}
    \int \mathcal{D} w_{AB}^{(+)} \int \mathcal{D} w_{AB}^{(-)} \\
    &\times
    \exp \left(
      -\frac{\rho_0}{\chi} \int d \mathbf{r} w_{AB}^{(+)}(\mathbf{r})^2
      - i\int d \mathbf{r} \hat{\rho}_+(\mathbf{r}) w_{AB}^{(+)}(\mathbf{r})
    \right) \\
    &\times
    \exp \left(
      - \frac{\rho_0}{\chi} \int d \mathbf{r} w_{AB}^{(-)}(\mathbf{r})^2
      +  \int d \mathbf{r} \hat{\rho}_-(\mathbf{r}) w_{AB}^{(-)}(\mathbf{r})
    \right)
\end{align*}
and
\begin{align*}
  \exp(- \beta U_2) =&
    \frac{1}{\Omega_+}
    \int \mathcal{D} w_+
    \exp \left(
      -\frac{\rho_0}{2\kappa}
      \int d \mathbf{r} w_+(\mathbf{r})^2
      + i\int d \mathbf{r} (\rho_0 - \hat{\rho}_+(\mathbf{r})) w_+(\mathbf{r})
    \right) \\
\end{align*}
Now the canonical partition function looks like
\begin{align*}
  Z_C =& \frac{1}{n_D!n_P! \left( \lambda_T^3 \right)^{n_D+n_P}}
    \frac{1}{\Omega_+ \Omega_{AB}^{(+)} \Omega_{AB}^{(-)} }
    \int \mathcal{D} w_+
    \int \mathcal{D} w_{AB}^{(+)}
    \int \mathcal{D} w_{AB}^{(-)} \\
    &\times
    \exp \left(
      - \frac{\rho_0}{2\kappa} \int d \mathbf{r} w_+(\mathbf{r})^2
      + i \rho_0 \int d\mathbf{r} w_+
      - \frac{\rho_0}{\chi} \int d \mathbf{r} w_{AB}^{(+)} (\mathbf{r})^2
      - \frac{\rho_0}{\chi} \int d \mathbf{r} w_{AB}^{(-)} (\mathbf{r})^2
    \right) \\
    &\times \int d \mathbf{r}^{n_DN_D} \int d \mathbf{r}^{n_P}
    \exp \left(
      - \sum_{i=1}^{n_D}    \sum_{j=1}^{P-1}
      \frac{3 \left| \mathbf{r}_{i,j+1} - \mathbf{r}_{i,j} \right| ^ 2 }
           { 2 b^2 }
      + \int d \mathbf{r} \left( -i w_+ \hat{\rho}_+
                                 -i w_{AB}^{(+)} \hat{\rho}_+
                                 +  w_{AB}^{(-)} \hat{\rho}_- \right)
    \right)
\end{align*}
Using the definitions of $\hat{\rho}_\pm$, and defining
  $w_A= i w_+ + i w_{AB}^{(+)} - w_{AB}^{(-)} $ and
  $w_B= i w_+ + i w_{AB}^{(+)} + w_{AB}^{(-)} $ we can rewrite
\begin{align*}
  \exp \left(
    \int d \mathbf{r} \left( -i w_+ \hat{\rho}_+
                             -i w_{AB}^{(+)} \hat{\rho}_+
                             +  w_{AB}^{(-)} \hat{\rho}_- \right)
  \right)
\end{align*}
as
\begin{align*}
  \prod_{j}^{n_{D}N_{DA}}
  \exp \left( -\omega_A(\mathbf{r}_j) \right)
  \prod_{k}^{n_{D}N_{DB}}
  \exp \left( -\omega_B(\mathbf{r}_k) \right).
  \prod_{l}^{n_P}
  \exp \left( -\omega_P(\mathbf{r}_l) \right).
\end{align*}
where $\omega_A$ and $\omega_B$ are defined as
\begin{align*}
  \omega_K(\mathbf{r}) = (h \ast w_K)(\mathbf{r})
\end{align*}
and $\omega_P$ is defined as
\begin{align*}
  \omega_P(\mathbf{r}) = (\Gamma \ast w_A)(\mathbf{r})
\end{align*}
Additionally, defining the bond transition probability $\Phi$ as
\begin{align*}
  \Phi(\mathbf{r} - \mathbf{r}^\prime) =
    \left( \frac{3}{2\pi b^2} \right) ^ {d/2}
    \exp \left( \frac{-3| \mathbf{r} - \mathbf{r}^\prime |^2}{2b^2} \right),
\end{align*}
we can rewrite the canonical partition function equation as
\begin{align*}
  Z_C =& \frac{1}{n_D!n_P! \left( \lambda_T^3 \right)^{n_D+n_P}}
    \frac{1}{\Omega_+ \Omega_{AB}^{(+)} \Omega_{AB}^{(-)} }
    \int \mathcal{D} w_+
    \int \mathcal{D} w_{AB}^{(+)}
    \int \mathcal{D} w_{AB}^{(-)} \\
    &\times
    \exp \left(
      - \frac{\rho_0}{2\kappa} \int d \mathbf{r} w_+(\mathbf{r})^2
      + i \rho_0 \int d\mathbf{r} w_+
      - \frac{\rho_0}{\chi} \int d \mathbf{r} w_{AB}^{(+)} (\mathbf{r})^2
      - \frac{\rho_0}{\chi} \int d \mathbf{r} w_{AB}^{(-)} (\mathbf{r})^2
    \right) \\
    &\times \int d \mathbf{r}^{n_DN_D} \int d \mathbf{r}^{n_P}
      \prod_j^{n_D} \prod_k^{N_D-1}
      \Phi(\mathbf{r}_{j,k+1} - \mathbf{r}_{j,k}) \\
    &\times
    \prod_{j}^{n_{D}N_{DA}} \exp \left( -\omega_A(\mathbf{r}_j) \right)
    \prod_{k}^{n_{D}N_{DB}} \exp \left( -\omega_B(\mathbf{r}_k) \right).
    \prod_{l}^{n_P}         \exp \left( -\omega_P(\mathbf{r}_l) \right)
\end{align*}
Then, we define $Q_D$ as
\begin{align*}
  Q_D = \frac{1}{V} \int d\mathbf{r} q_D (N_D, \mathbf{r})
\end{align*}
where
\begin{align*}
  q_D(j+1, \mathbf{r}) = \exp(-\omega_{X_{j+1}}(\mathbf{r}))
    \int d \mathbf{r}^\prime \Phi(\mathbf{r} - \mathbf{r}^\prime)
    q(j, \mathbf{r})
\end{align*}
where $X_{j+1}$ is either A or B depending on type of segment $j+1$
  and $q_D(1, \mathbf{r}) = \exp(-\omega_A(\mathbf{r}))$.
We also define $Q_P$ as
\begin{align*}
  Q_P = \frac{1}{V} \int d\mathbf{r} \exp (-\omega_P)
\end{align*}
With these definitions, we get
\begin{align*}
  Z_C =& \frac{V^{n_D+n_P}}
              {n_D!n_P! \left( \lambda_T^3 \right)^{n_D+n_P}}
    \frac{1}{\Omega_+ \Omega_{AB}^{(+)} \Omega_{AB}^{(-)} }
    \left( \frac{2\pi b^2}{3} \right)
      ^{(d/2)n_D(N_D-1)}
    \int \mathcal{D} w_+ \iint \mathcal{D} w_{AB}^{(\pm)} \\
    &\times \exp \left(
      - \frac{\rho_0}{2\kappa} \int d \mathbf{r} w_+(\mathbf{r})^2
      + i \rho_0 \int d\mathbf{r} w_+
      - \frac{\rho_0}{\chi} \int d \mathbf{r} w_{AB}^{(+)} (\mathbf{r})^2
      - \frac{\rho_0}{\chi} \int d \mathbf{r} w_{AB}^{(-)} (\mathbf{r})^2
    \right) \\
    &Q_D^{n_D} Q_P^{n_P}
\end{align*}
We can rewrite this as
\begin{align*}
  Z_C =& \frac{V^{n_D+n_P}}
              {n_D!n_P! \left( \lambda_T^3 \right)^{n_D+n_P}}
    \frac{1}{\Omega_+ \Omega_{AB}^{(+)} \Omega_{AB}^{(-)} }
    \left( \frac{2\pi b^2}{3} \right)
      ^{(d/2)n_D(N_D-1)} \\
    &\int \mathcal{D} w_+
    \iint \mathcal{D} w_{AB}^{(\pm)}
    \exp \left(
      -\mathcal{H}[w_+, w_{AB}^{(\pm)}]
    \right)
\end{align*}
where
\begin{align*}
  \mathcal{H}[w_+, w_{AB}^{(\pm)}] =&
    \frac{\rho_0}{2\kappa} \int d \mathbf{r} w_+(\mathbf{r})^2
    - i \rho_0 \int d\mathbf{r} w_+
    + \frac{\rho_0}{\chi} \int d \mathbf{r} w_{AB}^{(+)}(\mathbf{r})^2
    + \frac{\rho_0}{\chi} \int d \mathbf{r} w_{AB}^{(-)}(\mathbf{r})^2 \\
    &- n_D \log Q_D - n_P \log Q_P
\end{align*}

\section{Grand Canonical Derivation}

The grand canonical partition function is then given by
\begin{align*}
  Z_G(\mu_{D}, \mu_P, V, T) =
    \sum_{n_D}^\infty \exp(\beta\mu_D n_D)
    \sum_{n_P}^\infty \exp(\beta\mu_P n_P)
    Z_C(n_D, n_P, V, T)
\end{align*}
Now let's define activities $z_D$ and $z_P$ as
\begin{align*}
  z_K =
    z_{K0}
    \exp \left(\beta \mu_K \right)
\end{align*}
where
\begin{align*}
  z_{D0} =
    \frac{1}{\lambda_T^{3}}
    \left( \frac{2\pi b^2}{3} \right)^{d/2(N_D-1)}
\end{align*}
and
\begin{align*}
  z_{P0} =
    \frac{1}{\lambda_T^{3}}
\end{align*}
Now we can rewrite equation $Z_G$ as
\begin{align*}
  Z_G(\mu_D, \mu_P, V, T) =&
    \frac{1}{\Omega_+ \Omega_{AB}^{(+)} \Omega_{AB}^{(-)} }
    \int \mathcal{D} w_+ \iint \mathcal{D} w_{AB}^{(\pm)} \\
    &\times \exp \left(
      - \frac{\rho_0}{2\kappa} \int d \mathbf{r} w_+(\mathbf{r})^2
      + i \rho_0 \int d\mathbf{r} w_+
      - \frac{\rho_0}{\chi} \int d \mathbf{r} w_{AB}^{(+)} (\mathbf{r})^2
      - \frac{\rho_0}{\chi} \int d \mathbf{r} w_{AB}^{(-)} (\mathbf{r})^2
    \right) \\
    &\times \sum_{n_D}^\infty \frac{(z_DVQ_D)^{n_D}}{n_D!}
    \sum_{n_P}^\infty \frac{(z_PVQ_P)^{n_P}}{n_P!} \\
    =& \frac{1}{\Omega_+ \Omega_{AB}^{(+)} \Omega_{AB}^{(-)} }
    \int \mathcal{D} w_+
    \iint \mathcal{D} w_{AB}^{(\pm)} \\
    &\times \exp \left(
      - \frac{\rho_0}{2\kappa} \int d \mathbf{r} w_+(\mathbf{r})^2
      + i \rho_0 \int d\mathbf{r} w_+
      - \frac{\rho_0}{\chi} \int d \mathbf{r} w_{AB}^{(+)} (\mathbf{r})^2
      - \frac{\rho_0}{\chi} \int d \mathbf{r} w_{AB}^{(-)} (\mathbf{r})^2
    \right) \\
    &\times \exp (z_DVQ_D) \exp (z_PVQ_P)
\end{align*}
And finally, we get
\begin{align*}
  Z_G(\mu_D, \mu_P, V, T) =&
    \frac{1}{\Omega_+ \Omega_{AB}^{(+)} \Omega_{AB}^{(-)} }
    \int \mathcal{D} w_+ \iint \mathcal{D} w_{AB}^{(\pm)}
    \exp \left( - \mathcal{H}_G \left[w_+, w_{AB}^{(\pm)} \right] \right)
\end{align*}
where
\begin{align*}
  \mathcal{H}_G[w_+, w_{AB}^{(\pm)}] =&
    \frac{\rho_0}{2\kappa} \int d \mathbf{r} w_+(\mathbf{r})^2
    - i \rho_0 \int d\mathbf{r} w_+
    + \frac{\rho_0}{\chi} \int d \mathbf{r} w_{AB}^{(+)}(\mathbf{r})^2
    + \frac{\rho_0}{\chi} \int d \mathbf{r} w_{AB}^{(-)}(\mathbf{r})^2 \\
    &- z_D V Q_D - z_P V Q_P
\end{align*}

\section{Canonical 1S Update Derivation}

\subsection{$w_+$ Field}

First let's do the $w_+$ update derivation for the Canonical Ensemble.
\begin{align*}
  w_+^{t+1} =
    w_+^t - \lambda \left[
      \frac{\delta \mathcal{H}}{\delta w_+^t}
      + \left(  \frac{\delta \mathcal{H}}{\delta w_+^{t+1}} \right) _{lin}
      - \left(  \frac{\delta \mathcal{H}}{\delta w_+^{t}} \right) _{lin}
    \right]
\end{align*}
Taking the Fourier Transform,
\begin{align*}
  \hat{w}_+^{t+1} =
    \hat{w}_+^t - \lambda \left[
      \hat{\frac{\delta \mathcal{H}}{\delta w_+^t}}
      + \left( \hat{ \frac{\delta \mathcal{H}}{\delta w_+^{t+1}}} \right) _{lin}
      - \left( \hat{ \frac{\delta \mathcal{H}}{\delta w_+^{t}}} \right) _{lin}
    \right]
\end{align*}
Then, after plugging in the correct expressions, we can solve for
  $\hat{w}_+^{t+1}$ and take the inverse Fourier Transform to get $w_+^{t+1}$.
\begin{align*}
  \frac{\delta \mathcal{H}}{\delta w_+^t} &=
    \frac{\rho_0}{\kappa} w_+^t
    - i\rho_0
    + i [ (\rho_{DA,c} \ast h)(\mathbf{r})
          + (\rho_{DB,c} \ast h)(\mathbf{r})
          + (\rho_{P,c} \ast \Gamma)(\mathbf{r}) ] \\
  \hat{\frac{\delta \mathcal{H}}{\delta w_+^t}} &=
    \frac{\rho_0}{\kappa} \hat{w}_+^t
    - i \rho_0 \delta(\mathbf{k})
    + i [ \hat{\rho}_{DA,c} \hat{h}
          + \hat{\rho}_{DB,c} \hat{h}
          + \hat{\rho}_{P,c}  \hat{\Gamma} ] \\
  \left( \hat{\frac{\delta \mathcal{H}}{\delta w_+^t}} \right) _{lin} &=
    \frac{\rho_0}{\kappa} \hat{w}_+^t
    - i \hat{h}^2 \phi_D \rho_0 N_D
      (\hat{g}_{AA} + 2 \hat{g}_{AB} + \hat{g}_{BB}) i \hat{w}_+^t
    - i \phi_P \rho_0 \hat{\Gamma}^2 i \hat{w}_+^t \\
  &= \frac{\rho_0}{\kappa} \hat{w}_+^t
    + \hat{h}^2 \phi_D \rho_0 N_D
      (\hat{g}_{AA} + 2 \hat{g}_{AB} + \hat{g}_{BB}) \hat{w}_+^t
    + \phi_P \rho_0 \hat{\Gamma}^2 \hat{w}_+^t
\end{align*}
Assembling the pieces, we get
\begin{align*}
  \hat{w}_+^{t+1} =
    \hat{w}_+^t - \lambda \left[
      \frac{\rho_0}{\kappa} \hat{w}_+^t
      - i \rho_0 \delta(\mathbf{k})
      + i ( \hat{\rho}_{DA,c} \hat{h}
            + \hat{\rho}_{DB,c} \hat{h}
            + \hat{\rho}_{P,c}  \hat{\Gamma} )
      + A (\hat{w}_+^{t+1} - \hat{w}_+^t)
    \right]
\end{align*}
where
\begin{align*}
  A &=
  \frac{1}{w_+^t}
  \left( \hat{\frac{\delta \mathcal{H}}{\delta w_+^t}} \right) _{lin}
  =
  \frac{1}{w_+^{t+1}}
  \left( \hat{\frac{\delta \mathcal{H}}{\delta w_+^{t+1}}} \right) _{lin} \\
  &=
  \frac{\rho_0}{\kappa}
      + \hat{h}^2 \phi_D \rho_0 N_D
        (\hat{g}_{AA} + 2 \hat{g}_{AB} + \hat{g}_{BB})
      + \phi_P \rho_0 \hat{\Gamma}^2
\end{align*}
If we also let $B$ and $F$ equal
\begin{align*}
  B &= A - \frac{\rho_0}{\kappa} \\
    &= \hat{h}^2 \phi_D \rho_0 N_D
        (\hat{g}_{AA} + 2 \hat{g}_{AB} + \hat{g}_{BB})
      + \phi_P \rho_0 \hat{\Gamma}^2 \\
  F &= - i \rho_0 \delta (\mathbf{k})
       + i ( \hat{\rho}_{DA,c} \hat{h}
            + \hat{\rho}_{DB,c} \hat{h}
            + \hat{\rho}_{P,c}  \hat{\Gamma} )
\end{align*}
Then
\begin{align*}
  \hat{w}_+^{t+1} ( 1 + \lambda A ) =
  \hat{w}_+^t - \lambda \left( F - B \hat{w}_+^t \right) \\
  \hat{w}_+^{t+1} =
  \frac{\hat{w}_+^t - \lambda \left( F - B \hat{w}_+^t \right)}
       {1 + \lambda A}
\end{align*}

\subsection{$w_{AB}^{(+)}$}

For the $w_{AB}^{(+)}$ field, the relevant expressions are:
\begin{align*}
  \left. \frac{\delta \mathcal{H}}{\delta  w_{AB}^{(+)} } \right|_t &=
    \frac{2\rho_0}{\chi} \left. w_{AB}^{(+)} \right|_t
    + i [ (\rho_{DA,c} \ast h)(\mathbf{r})
          + (\rho_{DB,c} \ast h)(\mathbf{r})
          + (\rho_{P,c} \ast \Gamma)(\mathbf{r}) ] \\
    \left. \hat{\frac{\delta \mathcal{H}}{\delta w_{AB}^{(+)}}} \right|_t &=
    \frac{2\rho_0}{\chi} \left. \hat{w}_{AB}^{(+)} \right|_t
    + i [ \hat{\rho}_{DA,c} \hat{h}
          + \hat{\rho}_{DB,c} \hat{h}
          + \hat{\rho}_{P,c}  \hat{\Gamma} ] \\
  \left. \hat{\frac{\delta \mathcal{H}}{\delta w_{AB}^{(+)}}} \right| ^{lin}_t &=
    \frac{2\rho_0}{\chi} \left. \hat{w}_{AB}^{(+)} \right|_t
    - i \hat{h}^2 \phi_D \rho_0 N_D
      (\hat{g}_{AA} + 2 \hat{g}_{AB} + \hat{g}_{BB}) i \left.
        \hat{w}_{AB}^{(+)} \right|_t
    - i \phi_P \rho_0 \hat{\Gamma}^2 i \left. \hat{w}_{AB}^{(+)} \right|_t \\
  &= \frac{2\rho_0}{\chi} \left. \hat{w}_{AB}^{(+)} \right|_t
    + \hat{h}^2 \phi_D \rho_0 N_D
      (\hat{g}_{AA} + 2 \hat{g}_{AB} + \hat{g}_{BB})
        \left. \hat{w}_{AB}^{(+)} \right|_t
    + \phi_P \rho_0 \hat{\Gamma}^2 \left. \hat{w}_{AB}^{(+)} \right|_t
\end{align*}
This gives us
\begin{align*}
  \left. \hat{w}_{AB}^{(+)} \right|_{t+1} =
    \left. \hat{w}_{AB}^{(+)} \right|_t - \lambda \left[
      \frac{2\rho_0}{\chi} \left. \hat{w}_{AB}^{(+)} \right|_t
      + i ( \hat{\rho}_{DA,c} \hat{h}
            + \hat{\rho}_{DB,c} \hat{h}
            + \hat{\rho}_{P,c}  \hat{\Gamma} )
      + A ( \left. \hat{w}_{AB}^{(+)} \right|_{t+1}
            - \left. \hat{w}_{AB}^{(+)} \right|_t)
    \right]
\end{align*}
where
\begin{align*}
  A &=
  \frac{2\rho_0}{\chi}
      + \hat{h}^2 \phi_D \rho_0 N_D
        (\hat{g}_{AA} + 2 \hat{g}_{AB} + \hat{g}_{BB})
      + \phi_P \rho_0 \hat{\Gamma}^2
\end{align*}
If we let $B$ and $F$ equal
\begin{align*}
  B &= A - \frac{2\rho_0}{\chi} \\
    &= \hat{h}^2 \phi_D \rho_0 N_D
        (\hat{g}_{AA} + 2 \hat{g}_{AB} + \hat{g}_{BB})
      + \phi_P \rho_0 \hat{\Gamma}^2 \\
  F &= + i ( \hat{\rho}_{DA,c} \hat{h}
            + \hat{\rho}_{DB,c} \hat{h}
            + \hat{\rho}_{P,c}  \hat{\Gamma} )
\end{align*}
Then
\begin{align*}
  \left. \hat{w}_{AB}^{(+)} \right|_{t+1} ( 1 + \lambda A ) =
    \left. \hat{w}_{AB}^{(+)} \right|_t
    - \lambda \left( F - B \left. \hat{w}_{AB}^{(+)} \right|_t \right) \\
  \left. \hat{w}_{AB}^{(+)} \right|_{t+1} =
  \frac{\left. \hat{w}_{AB}^{(+)} \right|_t - \lambda
          \left( F - B \left. \hat{w}_{AB}^{(+)} \right|_t \right)}
       {1 + \lambda A}
\end{align*}

\subsection{$w_{AB}^{(-)}$ Field}

For the $w_{AB}^{(-)}$ field, the relevant expressions are:
\begin{align*}
  \left. \frac{\delta \mathcal{H}}{\delta w_{AB}^{(-)}} \right|_t &=
    \frac{2 \rho_0}{\chi} \left. w_{AB}^{(-)} \right|_t
    - (\rho_{DA,c} \ast h)(\mathbf{r})
    + (\rho_{DB,c} \ast h)(\mathbf{r})
    - (\rho_{P,c} \ast \Gamma)(\mathbf{r}) \\
  \left. \hat{\frac{\delta \mathcal{H}}{\delta w_{AB}^{(-)}}} \right|_t &=
    \frac{2 \rho_0}{\chi} \left. \hat{w}_{AB}^{(-)} \right|_t
    - \hat{\rho}_{DA,c} \hat{h}
    + \hat{\rho}_{DB,c} \hat{h}
    - \hat{\rho}_{P,c}  \hat{\Gamma} \\
  \left.
    \hat{\frac{\delta \mathcal{H}}{\delta w_{AB}^{(-)}}}
  \right| ^{lin}_t &=
    \frac{2\rho_0}{\chi} \left. \hat{w}_{AB}^{(-)} \right|_t
\end{align*}
Note that we don't do the weak inhomogeneity expansion here because the
  $w_{AB}^{(-)}$ field tends to be much less stiff than the $w_+$ fields and so
  doesn't need the extra approximation.
Now we get
\begin{align*}
  \left. \hat{w}_{AB}^{(-)} \right|_{t+1} &=
    \left. \hat{w}_{AB}^{(-)} \right|_t - \lambda \left[
      \frac{2\rho_0}{\chi} \left. \hat{w}_{AB}^{(-)} \right|_t
      - \hat{\rho}_{DA,c} \hat{h}
      + \hat{\rho}_{DB,c} \hat{h}
      - \hat{\rho}_{P,c}  \hat{\Gamma}
      + \frac{2\rho_0}{\chi}
        ( \left. \hat{w}_{AB}^{(-)}\right|_{t+1}
          - \left. \hat{w}_{AB}^{(-)} \right|_t
        )
    \right] \\
  \left. \hat{w}_{AB}^{(-)} \right|_{t+1} ( 1 + \lambda \frac{2 \rho_0}{\chi} ) &=
    \left. \hat{w}_{AB}^{(-)} \right|_t - \lambda \left(
      - \hat{\rho}_{DA,c} \hat{h}
      + \hat{\rho}_{DB,c} \hat{h}
      - \hat{\rho}_{P,c}  \hat{\Gamma}
    \right) \\
  \left. \hat{w}_{AB}^{(-)} \right|_{t+1} &=
    \frac{
      \left. \hat{w}_{AB}^{(-)} \right|_t - \lambda \left(
        - \hat{\rho}_{DA,c} \hat{h}
        + \hat{\rho}_{DB,c} \hat{h}
        - \hat{\rho}_{P,c}  \hat{\Gamma}
      \right)
    }
    {
      \left( 1 + \lambda \frac{2 \rho_0}{\chi} \right)
    }
\end{align*}

\section{Calculating Densities}

\subsection{Canonical Ensemble}

In the Canonical Ensemble, the polymer densities are given by
\begin{align*}
  \rho_{DA,c} &=
    -n_D \frac{\delta \log Q_D}{\delta \omega_A(\mathbf{r})}
    = \frac{n_D}{V Q_D}
    \sum_{j=1}^{P_A}
    q_D(j, \mathbf{r})
    e^{\omega_A(\mathbf{r})}
    q_D^\dagger(P-j, \mathbf{r}) \\
  \rho_{DB,c} &=
    -n_D \frac{\delta \log Q_D}{\delta \omega_B(\mathbf{r})}
    = \frac{n_D}{V Q_D}
    \sum_{j=P_A+1}^{P}
    q_D(j, \mathbf{r})
    e^ {\omega_B(\mathbf{r})}
    q_D^\dagger(P-j, \mathbf{r})
\end{align*}
and the particle density is given by
\begin{align*}
  \rho_P(\mathbf{r}) =
    -n_P \frac{\delta \log Q_P}{\delta \omega_P(\mathbf{r})}
    = \frac{n_P}{V Q_P} e^{-\omega_P(\mathbf{r})}
\end{align*}

\subsection{Grand Canonical Ensemble}

In the Grand Canonical Ensemble, the polymer densities are given by
\begin{align*}
  \rho_{DA,c} &=
    -z_D V \frac{\delta Q_D}{\delta \omega_A(\mathbf{r})}
    = z_D
    \sum_{j=1}^{P_A}
    q_D(j, \mathbf{r})
    e^{\omega_A(\mathbf{r})}
    q_D^\dagger(P-j, \mathbf{r}) \\
  \rho_{DB,c} &=
    -z_D \frac{\delta Q_D}{\delta \omega_B(\mathbf{r})}
    = z_D
    \sum_{j=P_A+1}^{P}
    q_D(j, \mathbf{r})
    e^ {\omega_B(\mathbf{r})}
    q_D^\dagger(P-j, \mathbf{r})
\end{align*}
and the particle density is given by
\begin{align*}
  \rho_P(\mathbf{r}) =
    -z_P \frac{\delta Q_P}{\delta \omega_P(\mathbf{r})}
    = z_P e^{-\omega_P(\mathbf{r})}
\end{align*}

\section{Comparing $z_D$ values with 2 Field Model}

From the 2-field model derivation,
\begin{align*}
  z_{D0,2} = \frac{1}{\lambda_T^3}
  \exp \left( -\frac{N_D\chi}{4}  \right)
  \left( \frac{2\pi b^2}{3} \right)^{d/2(N_D-1)} 
\end{align*}
And from the 3-field model derivation, (this document)
\begin{align*}
  z_{D0,3} = \frac{1}{\lambda_T^3}
  \left( \frac{2\pi b^2}{3} \right)^{d/2(N_D-1)} 
\end{align*}
Therefore, to compare 2-field and 3-field simulations, we need to take this
  into account to make sure $\mu_D$ matches between them.
Thus, given a 3-field model using $z_{D,3}$, the corresponsing value of
  $z_{D,2}$ necessary to match a 2-field model is
\begin{align*}
  z_{D,2} = \exp \left( -\frac{N_D\chi}{4}  \right) z_{D,3}
\end{align*}

\section{Simplification for Homogeneous System}

In a homogeneous system, the fields $w_+$, $w_{AB}^{(+)}$, and $w_{AB}^{(-)}$
  are all constants.
To simplify solving the equations, it is convenient to use only real numbers by
  solving for $iw_+$, $iw_{AB}^{(+)}$, and $w_{AB}^{(-)}$.
With this in mind, we can rewrite the Grand Canonical Hamiltonian as

\begin{align*}
  H_G = - &\frac{\rho_0 V}{2\kappa} (i w_+)^2
        - \rho_0 V (i w_+)
        - \frac{\rho_0 V}{\chi} (i w_{AB}^{(+)})^2
        + \frac{\rho_0 V}{\chi} (w_{AB}^{(-)})^2
        - z_D V Q_D
        - z_P V Q_P
\end{align*}

If we solve with simple Euler equations, we get

\begin{align*}
  iw_+^{t+1} &= iw_+^t - i\lambda_+ \frac{\partial H_G/V}{\partial w_+^t} \\
    &= iw_+^t + \lambda_+ \frac{\partial H_G/V}{\partial iw_+^t} \\
  \left. iw_{AB}^{(+)} \right|_{t+1} &= \left. iw_{AB}^{(+)} \right|_t
    - i\lambda_+ \left. \frac{\partial H_G/V}{\partial w_{AB}^{(+)}} \right|_t \\
    &= \left. iw_{AB}^{(+)} \right|_t
    + \lambda_+ \left. \frac{\partial H_G/V}{\partial iw_{AB}^{(+)}} \right|_t \\
  \left. w_{AB}^{(-)} \right|_{t+1} &= \left. w_{AB}^{(-)} \right|_t
    - \lambda_- \left. \frac{\partial H_G/V}{\partial w_{AB}^{(-)}} \right|_t
\end{align*}

With this in mind, we can rewrite the partition functions as

\begin{align*}
  Q_D &= \exp(-P_A w_A - P_B w_B) \\
  Q_P &= \exp(-\rho_0 V_P w_A)
\end{align*}

Let's note that

\begin{align*}
  \rho_D &= \rho_{DA} + \rho_{DB} \\
  &= -z_D \frac{\partial Q_D}{\partial \omega_A}
     -z_D \frac{\partial Q_D}{\partial \omega_B} \\
  &= -z_D \frac{\partial Q_D}{\partial w_A}
     -z_D \frac{\partial Q_D}{\partial w_B} \\
  &= z_D P \exp(-P_A w_A - P_B w_B) \\
  &= z_D P Q_D
\end{align*}

and

\begin{align*}
  \rho_{P,c} &= -z_P \frac{\partial Q_P}{\partial \omega_P} \\
  &= z_P \exp(-\rho_0 V_P w_A) \\
  \rho_{P,c} &= z_P Q_P \\
  \rho_P &= \rho_0 V_P z_P Q_P
\end{align*}

with the following derivatives:

\begin{align*}
  \frac{\partial H_G/V}{\partial iw_+} &=
  - \frac{\rho_0}{\kappa} (iw_+)
  - \rho_0
  - \frac{2 \rho_0}{\chi} (iw_{AB}^{(+)})
  + \frac{2 \rho_0}{\chi} (w_{AB}^{(-)})
  - z_D \frac{\partial Q_D}{\partial iw_+}
  - z_P \frac{\partial Q_P}{\partial iw_+}
\end{align*}

With this in mind, we can rewrite the partition functions as

\begin{align*}
  Q_D &= \exp(-f P w_A - (1-f) P w_B) \\
  &= \exp(-f P (iw_+ + iw_{AB}^{(+)} - w_{AB}^{(-)})
          - (1-f) P (iw_+ + iw_{AB}^{(+)} + w_{AB}^{(-)})
         ) \\
  &= \exp(-P(iw_+ + iw_{AB}^{(+)}) + (2f-1)P w_{AB}^{(-)})
\end{align*}

and \begin{align*}
  Q_P &= \exp(-\rho_0 V_P w_A) \\
      &= \exp(-\rho_0 V_P (iw_+ + iw_{AB}^{(+)} - w_{AB}^{(-)}))
\end{align*}

Then we can rewrite the Grand Canonical Hamiltonian as

\begin{align*}
  H_G = - &\frac{\rho_0 V}{2\kappa} (i w_+)^2
        - \rho_0 V (i w_+)
        - \frac{\rho_0 V}{\chi} (i w_{AB}^{(+)})^2
        + \frac{\rho_0 V}{\chi} (w_{AB}^{(-)})^2 \\
        - &z_D V \exp(-P(iw_+ + iw_{AB}^{(+)}) + (2f-1)P w_{AB}^{(-)}) \\
        - &z_P V \exp(-\rho_0 V_P (iw_+ + iw_{AB}^{(+)} - w_{AB}^{(-)}))
\end{align*}

If we solve with simple Euler equations, we get

\begin{align*}
  w_+^{t+1} &= w_+^t - \lambda_+ \frac{\partial H_G/V}{\partial w_+^t} \\
  \left. w_{AB}^{(+)} \right|_{t+1} &= \left. w_{AB}^{(+)} \right|_t
    - \lambda_+ \left. \frac{\partial H_G/V}{\partial w_{AB}^{(+)}} \right|_t \\
  \left. w_{AB}^{(-)} \right|_{t+1} &= \left. w_{AB}^{(-)} \right|_t
    - \lambda_- \left. \frac{\partial H_G/V}{\partial w_{AB}^{(-)}} \right|_t
\end{align*}

with the following derivatives:

\begin{align*}
  \frac{\partial H}{\partial w_+^t} &=
  - \frac{\rho_0}{\kappa} (iw_+)
  - \rho_0
  - \frac{2 \rho_0}{\chi} (iw_{AB}^{(+)})
  + \frac{2 \rho_0}{\chi} (w_{AB}^{(-)})
\end{align*}

\end{document}
